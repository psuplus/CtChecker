\section{Background}
\subsection{Information Flow Analysis}
Information flow analysis tracks interactions of information throughout a
program. A simple application of this analysis is used in LOMAC to restrict
access to data of various integrity levels \cite{fraser2000lomac}. LOMAC
maintains a concept of information integrity for each object, and restricts
information from flowing from low integrity objects to high integrity objects.
If a program source is inspected, and data is given a level of integrity or
confidentiality then by following the flow of information in the program, it is
possible to track which data have been computed from confidential or high
integrity information. In cryptographic systems, the encryption keys are
confidential, if other data within the program is calculated from the encryption
key, there is a flow of information from the confidential data to the computed
data \cite{wang2017cached}. Compilers also use information flow for optimization
purposes. By identifying what information is used and when, alterations to a
program can be made to improve cache locality, hide memory latency and improve
performance in other manners. For program level security, a compiler could be
able to check the level of integrity for any variable through the use of
information flow. By tracking confidential or sensitive pieces of information,
any variable or decision which is computed from a confidential data has a flow
from the confidential data to the variable or branch. An attacker may be able to
exploit parts of a program which are dependent on some confidential data, and
with sufficient flow, can possibly reconstruct the confidential data. An example
is given in figure \ref{alg:simpleflow} to illustrate this concern.

\begin{figure}
  \hrulefill
  \begin{algorithmic}
    \State $k =$ sensitive information
    \State $d.x =  k + 1$
    \State $d.y = 0$
    \State $a = -1$

    \If {$d.y == k/2$}
      \State $a = 4$ 
    \EndIf

    % \If {$d.x == 2$}
      % \State $b = 2$ 
    % \Endif
  \end{algorithmic}
  \hrulefill
  \caption{Example of Explicit and Implicit Flow}
  \label{alg:simpleflow}
\end{figure}

In this simple example $k$ is confidential data, through information flow we
can see that the value of $d.x$ is directly computed from $k$. The value $d.y$
however is not confidential due to its value having no reliance on the
confidential data. The flow between $k$ and $d.x$ is called explicit flow,
since the value of $d.x$ is computed from the confidential data. As the
analysis continues, any value which is computed from $d.x$ will also be
treated as confidential data. There is also implicit flow which can be seen in
the first branch on $d.y$ where it is compared with some value computed from
$k$. In this case $d.y$ is still considered non-confidential data, since the
value has no relation to $k$, but there is a flow from $k$ to $a$ indirectly.
Although, $a$ is not computed from $k$, after the branch executes, the value
of $a$ may or may not have changed. If it does change, the value of $k$ has been
determined. If it does not change, the value of $k$ is not determined, but a
possible value has been ruled out. This indirect flow from $k$ to $a$ is what
is called implicit flow. Tracking the flow of information between variables can
be done by creating a set of constraints for each instruction to represent the
direction of flow. Creating these constraints with mutable pointers, requires
more information than is provided from IR code from the compilation process.

Information flow analyses have been capable to find even hardware based
vulnerabilities from analyzing the source. Information flow has been used to
detect leaking of confidential data through the source and also leaking data
through hardware side channels such as through the cache
\cite{DBLP:journals/corr/DoychevK16}.

\subsection{Points-to Analysis}

A Points-to analysis is a reference tracking analysis to identify the targets of
pointers in a program. Information flow alone is not enough once pointers are
involved. For example, in Algorithm \ref{alg:simpleflow}, it must be possible to
identify from the source the location of $d$, and subsequently the location of
$x$. The points-to analysis provides this information so that the proper
location of $x$ and $y$ can be known even if the value of $d$ changes, meaning
it points to another location in memory. In addition to providing spatial
information, the points-to analysis used in this research, DSA,  provides type and data
layout information \cite{DSA-lattner}. The analysis accomplishes tracking by maintaining a internal
graph representation of memory throughout the program. The graph is made up of
nodes which are units of memory that are able to be the target of a pointer.
Data structures identified in program source have one node for an instance of
the data structure, where all fields pertaining to that data structure are
represented by that node. The edges in the graph are references, so it is
possible to identify where the target of a pointer resides. Additionally, this
analysis provides type information along with offsets of the type from the base
pointer. In the baseline tainted flow analysis, the granularity of the results
was achieved by creating  constraints of information flow between nodes from the
points-to graph. This however, as will be discussed, leads to an
over-approximation of the resulting tainted values. The points-to analysis used
also has a method of determining the validity of the type information. 

\subsection{Timing Channels in Cryptosystems}

A timing channel is a side channel in which an attacker uses execution time to
learn information about sensitive data. In some implementations of sliding
window exponentiation, the sequences of squares and multiplies could be measured
due to differences in each methods execution time. This attack though timing
based can be observed in source code in OpenSSL's previous implementation of
sliding window exponentiation, shown in Algorithm \ref{alg:timingsqrmlt}. 

\begin{figure}
\begin{lstlisting}
for (i = 1; i < bits; i++) {
    if (!BN_sqr(v, v, ctx))
        goto err;
    if (BN_is_bit_set(p, i)) {
        if (!BN_mul(rr, rr, v, ctx))
            goto err;
    }
}
\end{lstlisting}
\caption{Square and Multiply Timing Channel}
\label{alg:timingsqrmlt}
\end{figure}

The for-loop here iterates through the number of bits in the confidential power
\texttt{p}, each time through squaring \texttt{v} until a set bit is encountered
in \texttt{p}. Once a set bit is encountered, an additional multiply step is
executed before proceeding to the next loop iteration. This lends itself to a
timing attack due to multiplication more expensive to compute than a square. The
timing channel exists because the number of operations executed during one
iteration changes based on the value of the confidential data. The result is if
one can determine the locations of the multiplies and the length of time
expected between each multiplication, the key can be rebuilt as has been done
by Bernstein et al \cite{bernstein2017sliding}. There are other forms of timing
channels which may or may not be viewable in source code, such as disk access
timing or network timing attacks. 
