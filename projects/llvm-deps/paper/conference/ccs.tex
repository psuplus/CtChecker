\documentclass[sigconf, authorversion]{acmart}

\usepackage[utf8]{inputenc}
\usepackage{graphicx}
\usepackage{amsmath}
\usepackage{algorithm}
\usepackage{algpseudocode}
\usepackage{algorithmicx}
\usepackage{listings}
\usepackage{fixltx2e}
\usepackage{booktabs}
\usepackage{float}
\usepackage{url}
\usepackage{xcolor}
%testtk
\lstdefinestyle{codestyle}{
  basicstyle=\footnotesize\ttfamily,
  numberstyle=\tiny,
  numbers=left,
  xleftmargin=2em,
  framexleftmargin=1.5em,
  frame=lines,
  breaklines=true,
  tabsize=2,
  escapeinside={<@}{@>},
  captionpos=b,
  language=C++
}
\lstdefinestyle{nolinenumberstyle}{
  basicstyle=\footnotesize\ttfamily,
  numbers=none,
  xleftmargin=0em,
  framexleftmargin=0em,
  frame=none,
  breaklines=true,
  tabsize=2,
  captionpos=b,
  language=C++
}
\lstset{style=codestyle}
%\renewcommand*{\lstlistoflistings}{List of \lstlistingname s}

%% Commands
\newcommand{\codevar}[1]{\textit{#1}}
\newcommand{\codefn}[1]{\textit{#1}}
\newcommand{\codefile}[1]{\textit{#1}}
\newcommand{\ruleabove}{}
\newcommand{\rulebelow}{}

%Tables
\newcommand{\ra}[1]{\renewcommand{\arraystretch}{#1}}
\newcommand{\explicitflow}[1]{\vspace{-0.25cm}{\footnotesize\hspace{0.25cm}\textbf{Explicit:} #1 \par}}
\newcommand{\implicitflow}[1]{{\footnotesize\hspace{0.25cm}\textbf{Implicit:} #1 \par}}

\DeclareMathOperator{\pointer}{pointer}
\DeclareMathOperator{\result}{result}
\DeclareMathOperator{\ptrval}{ptrval}
\DeclareMathOperator{\offset}{offset}
\DeclareMathOperator{\myvalue}{value}

% TOG prefers author-name bib system with square brackets
\citestyle{acmnumeric}
\setcitestyle{numbers,sort&compress}


%
% \usepackage[ruled]{algorithm2e} % For algorithms
% \renewcommand{\algorithmcfname}{ALGORITHM}
% \SetAlFnt{\small}
% \SetAlCapFnt{\small}
% \SetAlCapNameFnt{\small}
% \SetAlCapHSkip{0pt}
% \IncMargin{-\parindent}

% Metadata Information
%\acmJournal{TOG}
\acmConference[CCS'19]{Computer and Communications Security}
\acmVolume{26}
\acmYear{2019}
\acmMonth{1}

% Copyright
\setcopyright{acmcopyright}
%\setcopyright{acmlicensed}
%\setcopyright{rightsretained}
%\setcopyright{usgov}
%\setcopyright{usgovmixed}
%\setcopyright{cagov}
%\setcopyright{cagovmixed}

% DOI
\acmDOI{0000001.0000001_2}

% Paper history
\received{February 2019}

\newcommand{\sysname}{CtChecker}

% Document starts
\begin{document}
% Title portion
\title{\sysname: a Highly Precise and Efficient Static Information Analysis for
Detecting Non-Constant Time Code}

\author{Adam Mohammed}
\affiliation{%
  \institution{The Pennsylvania State University}
}
\email{aqm5498@psu.edu}

\author{Ernest Defoy III}
\affiliation{%
  \institution{The Pennsylvania State University}
}
\email{edd5103@psu.edu}

\author{Danfeng Zhang}
\affiliation{%
  \institution{The Pennsylvania State University}
}
\email{zhang@psu.edu}

% \renewcommand\shortauthors{Zhou, G. et al}

\begin{abstract}
Timing channel attacks are emerging as real-world threats to computer security.
While cryptosystems have been the major targets of timing attacks, recent
attacks such as Spectre and Meltdown show that non-constant time code, when
combined with other attack vectors such as speculative execution, can reveal
arbitrary memory from the victim's process.  In cryptosystems, an effective
countermeasure against timing attacks is the constant-time programming
discipline. However, strictly enforcing the discipline manually is both time
consuming and error prone. Recent efforts on identifying non-constant code
fragments verify a 2-property, meaning that with identical public inputs, two
runs of a program will produce identical control flows, memory accesses, etc.
However, at least implicitly, such tools consider 2 runs of the program,
resulting in either poor performance or key information (e.g., loop invariants)
across 2 runs being lost.

In this paper, we build \sysname, a static sound analysis for identifying
non-constant code. Under the hood, \sysname{} verifies a 1-property, namely
program dependence, making it more efficient than existing constant-time
verifiers. For precision, \sysname{} is both field-sensitive and
context-sensitive, and it supports declassification. Somewhat surprisingly,
\sysname{} reports fewer false positives in cryptosystems compared with
existing tools, since the latter suffers from lost information such as loop
invariants. Evaluation on multiple real-world cryptosystems shows that
\sysname{} analyzes over 40K lines of source code in half an hour, and its
false-positive rate is low.
\end{abstract}


%
% The code below should be generated by the tool at
% http://dl.acm.org/ccs.cfm
% Please copy and paste the code instead of the example below.
%
 \begin{CCSXML}
<ccs2012>
<concept>
<concept_id>10002978.10003006.10011608</concept_id>
<concept_desc>Security and privacy~Information flow control</concept_desc>
<concept_significance>500</concept_significance>
</concept>
</ccs2012>
\end{CCSXML}

\ccsdesc[500]{Security and privacy~Information flow control}
%
% End generated code
%


\keywords{}



\maketitle

\section{Introduction}

Timing channels can be found in modern cryptographic systems that are widely
adopted \cite{kochertiming,brumley2005remote,percival2005cache,bernstein2017sliding}.
These widely adopted systems are used to provide confidentiality and integrity
of data communicated between parties. Attackers can exploit these timing
channels to compromise the assumptions of these safe communication methods.
Timing channels exist in many forms and are often the result of implementation
details which are not available in theoretical proposals of a protocol.

In cryptosystems, an effective countermeasure against timing attacks is the
constant-time programming discipline.  While a strict enforcement of this
discipline will likely rule out timing attacks, doing so is also infeasible for
a couple of reasons: (1) identifying non-constant time code  is often an
error-prone and manual process, (2) real-world code typically contains
intentional non-constant fragments that does not leak crucial information.

Many methods have emerged as a response to attempt to automate the process of
identifying timing channels. Currently as vulnerabilities are
identified, the measures taken to mitigate these vulnerabilities are highly
specific to the problem. It is important to be able to analyze software to
identify possible timing channels, and mitigate them as necessary. The problem
becomes identifying the vulnerabilities. Static analyses exist to be able to
identify these vulnerabilities.
  
  Static analyses can struggle with the number of false positives being fairly
  high. The precision is often sacrificed to decrease the runtime of an
  analysis. Practical applications of a static analysis would optimally have the
  least amount of false positives while remaining sound. Developers and
  engineers who design and implement these libraries can then more easily focus
  on the implementations most likely to be vulnerable.
  
  There are many types of static analysis targeted at identifying timing
  channels \cite{cached-zhang, brotzmancasym, molnar2005program}. There have also been
  works which analyze cryptographic protocols using proof-engineering techniques
  \cite{proof-engineering}. The issue with these approaches is in terms of
  scalability and accurate representation of system state. Static analysis tools
  have the ability to work directly with the binary so the state is closer to
  the actual state during execution of the software.

  This work is based on a static analysis which tracked information-flow in
  order to find variables computed from secret values as well as find
  secret-dependent branches \cite{moore2011static}. The analysis uses
  \textit{DSA}, a points-to analysis to handle dynamically allocated memory
  \cite{DSA-lattner}. The information-flow analysis is heavily reliant on the
  points-to analysis to track the information flow of practical programs. The
  precision of the information-flow analysis is influenced by how it integrates
  with the points-to analysis.
  
  The static information flow analysis looks at the effects of adding extra
  features to the baseline information-flow analysis to increase precision. The
  positives that are eliminated should not be any high-risk positives and the
  total number should be fewer than the baseline. In the case study, the
  analysis is run on modular exponentiation source code used in popular
  cryptosystems, with additional features. There were two features added to the
  baseline analysis to try and achieve this goal, i) field-sensitivity, ii)
  white-listing. Lastly, the analysis has to be conservative when source code
  for a given function call is not provided, so the analysis was tested with and
  without the full library source code.
  
  The field sensitivity is improved by relying on more information provided by
  the points-to analysis. The white-list is aimed at removing results that may
  actually be leaking information but are considered acceptable. The white-list
  can also be used to ignore results which cause other erroneous positives. 

  A case-study is performed by analyzing the modular exponentiation source code
  provided for Libgcrypt, OpenSSL, mbedTLS, and BearSSL. The results help
  identify what features are necessary to improve the precision of an
  information-flow analysis. For example, in Libgcrypt the number of results was
  reduced from 64 to 24 total results. Of the 64 baseline results there were 5
  classified as high-risk and in the improved analysis the high-risk results
  were still present. The reduced number of results came from eliminated false
  positives.



\section{Background}
\subsection{Information Flow Analysis}
Information flow analysis tracks interactions of information throughout a
program. A simple application of this analysis is used in LOMAC to restrict
access to data of various integrity levels \cite{fraser2000lomac}. LOMAC
maintains a concept of information integrity for each object, and restricts
information from flowing from low integrity objects to high integrity objects.
If a program source is inspected, and data is given a level of integrity or
confidentiality then by following the flow of information in the program, it is
possible to track which data have been computed from confidential or high
integrity information. In cryptographic systems, the encryption keys are
confidential, if other data within the program is calculated from the encryption
key, there is a flow of information from the confidential data to the computed
data \cite{wang2017cached}. Compilers also use information flow for optimization
purposes. By identifying what information is used and when, alterations to a
program can be made to improve cache locality, hide memory latency and improve
performance in other manners. For program level security, a compiler could be
able to check the level of integrity for any variable through the use of
information flow. By tracking confidential or sensitive pieces of information,
any variable or decision which is computed from a confidential data has a flow
from the confidential data to the variable or branch. An attacker may be able to
exploit parts of a program which are dependent on some confidential data, and
with sufficient flow, can possibly reconstruct the confidential data. An example
is given in figure \ref{alg:simpleflow} to illustrate this concern.

\begin{figure}
  \hrulefill
  \begin{algorithmic}
    \State $k =$ sensitive information
    \State $d.x =  k + 1$
    \State $d.y = 0$
    \State $a = -1$

    \If {$d.y == k/2$}
      \State $a = 4$ 
    \EndIf

    % \If {$d.x == 2$}
      % \State $b = 2$ 
    % \Endif
  \end{algorithmic}
  \hrulefill
  \caption{Example of Explicit and Implicit Flow}
  \label{alg:simpleflow}
\end{figure}

In this simple example $k$ is confidential data, through information flow we
can see that the value of $d.x$ is directly computed from $k$. The value $d.y$
however is not confidential due to its value having no reliance on the
confidential data. The flow between $k$ and $d.x$ is called explicit flow,
since the value of $d.x$ is computed from the confidential data. As the
analysis continues, any value which is computed from $d.x$ will also be
treated as confidential data. There is also implicit flow which can be seen in
the first branch on $d.y$ where it is compared with some value computed from
$k$. In this case $d.y$ is still considered non-confidential data, since the
value has no relation to $k$, but there is a flow from $k$ to $a$ indirectly.
Although, $a$ is not computed from $k$, after the branch executes, the value
of $a$ may or may not have changed. If it does change, the value of $k$ has been
determined. If it does not change, the value of $k$ is not determined, but a
possible value has been ruled out. This indirect flow from $k$ to $a$ is what
is called implicit flow. Tracking the flow of information between variables can
be done by creating a set of constraints for each instruction to represent the
direction of flow. Creating these constraints with mutable pointers, requires
more information than is provided from IR code from the compilation process.

Information flow analyses have been capable to find even hardware based
vulnerabilities from analyzing the source. Information flow has been used to
detect leaking of confidential data through the source and also leaking data
through hardware side channels such as through the cache
\cite{DBLP:journals/corr/DoychevK16}.

\subsection{Points-to Analysis}

A Points-to analysis is a reference tracking analysis to identify the targets of
pointers in a program. Information flow alone is not enough once pointers are
involved. For example, in Algorithm \ref{alg:simpleflow}, it must be possible to
identify from the source the location of $d$, and subsequently the location of
$x$. The points-to analysis provides this information so that the proper
location of $x$ and $y$ can be known even if the value of $d$ changes, meaning
it points to another location in memory. In addition to providing spatial
information, the points-to analysis used in this research, DSA,  provides type and data
layout information \cite{DSA-lattner}. The analysis accomplishes tracking by maintaining a internal
graph representation of memory throughout the program. The graph is made up of
nodes which are units of memory that are able to be the target of a pointer.
Data structures identified in program source have one node for an instance of
the data structure, where all fields pertaining to that data structure are
represented by that node. The edges in the graph are references, so it is
possible to identify where the target of a pointer resides. Additionally, this
analysis provides type information along with offsets of the type from the base
pointer. In the baseline tainted flow analysis, the granularity of the results
was achieved by creating  constraints of information flow between nodes from the
points-to graph. This however, as will be discussed, leads to an
over-approximation of the resulting tainted values. The points-to analysis used
also has a method of determining the validity of the type information. 

\subsection{Timing Channels in Cryptosystems}

A timing channel is a side channel in which an attacker uses execution time to
learn information about sensitive data. In some implementations of sliding
window exponentiation, the sequences of squares and multiplies could be measured
due to differences in each methods execution time. This attack though timing
based can be observed in source code in OpenSSL's previous implementation of
sliding window exponentiation, shown in Algorithm \ref{alg:timingsqrmlt}. 

\begin{figure}
\begin{lstlisting}
for (i = 1; i < bits; i++) {
    if (!BN_sqr(v, v, ctx))
        goto err;
    if (BN_is_bit_set(p, i)) {
        if (!BN_mul(rr, rr, v, ctx))
            goto err;
    }
}
\end{lstlisting}
\caption{Square and Multiply Timing Channel}
\label{alg:timingsqrmlt}
\end{figure}

The for-loop here iterates through the number of bits in the confidential power
\texttt{p}, each time through squaring \texttt{v} until a set bit is encountered
in \texttt{p}. Once a set bit is encountered, an additional multiply step is
executed before proceeding to the next loop iteration. This lends itself to a
timing attack due to multiplication more expensive to compute than a square. The
timing channel exists because the number of operations executed during one
iteration changes based on the value of the confidential data. The result is if
one can determine the locations of the multiplies and the length of time
expected between each multiplication, the key can be rebuilt as has been done
by Bernstein et al \cite{bernstein2017sliding}. There are other forms of timing
channels which may or may not be viewable in source code, such as disk access
timing or network timing attacks. 

\section{\sysname{} Design}

In this section, we describe the design details of \sysname{}.

In this work, adjustments were made to improve the precision of a baseline
taint analysis. The baseline analysis was proposed by Moore et al
\cite{moore2011static}. Adding field-sensitivity and adding a whitelist both
improved precision. Adding field-sensitivity aided in more
accurately representing flow within structures and arrays. The baseline could
not track flow to members of a structure, only flow to the structure itself. The
field-sensitive analysis is the major change to the baseline system. This
section will outline the three sections which make up the analysis. The
information flow capture parses the IR code to identify the values from which
information flows. The constraint generation is done by interpreting the
captured flows and creating constraints. Lastly, to identify secret-dependent
branches, the least solution is found and branch conditions are checked to see if
sensitive information was used to compute the condition.

\subsection{Generating Information Flows from IR Code}

   Given a program in the LLVM IR representation, flow of information is tracked
   by identifying the operands. Each operand can be considered as a source or a
   sink, though this alone is not enough information when considering the
   effects of pointers. Since pointers can point to different data and the data
   to which they point can change, the points-to analysis is used. The
   points-to analysis is used to keep track of which memory locations are
   associated with the LLVM IR values identified as sources or sinks.

   For each LLVM value, there are 3 elements which can be constrained. There are
   3 transformation functions which return the element to be constrained for
   that value. For any operation, zero or more of the three functions (V, D, R)
   may be used in generating a constraint for the instruction. Forming a
   constraint requires a source constraint element and a sink constraint
   element. The flows identified from the instructions create a constraint by
   adding a rule, which states that the source constraint element flows to the sink
   constraint element.

   \begin{itemize}
   \item $V(x)$ is the constraint element associated with the IR value $x$.
   \item $D(x, offset, size)$ is the constraint element with the appropriate
     type $size$ associated with the memory represented by IR value $x$ and the
     $offset$.
   \item $R(x, size)$ is the set of all reachable memory locations that are
     represented by IR value $x$.
   \end{itemize}

   The baseline analysis used these same transformations, but the addition
   of field sensitivity changed the behavior of the functions $D$ and $R$, and the
   memory nodes were only able to be constrained as a single entity.
   Since a whole data structure was referred to as a single element, tracking
   the flow to individual fields was not available. The \textit{DSA} analysis
   provides a type map, which can be used to identify the field type and offset
   within the data structure. Using this information, the $D$ and $R$ functions
   consumed an extra operand to return the element to be constrained for a
   particular field.

   The $R$ transformation is similar to the $D$ transformation in that it
   returns the constraint element relevant to the memory addressed, but the $R$
   transformation is used in the case of external functions or source. The $R$
   transformation is the more conservative transformation, returning all possible
   constraint elements tied to a pointer, so the offset and size are not
   required like they are for the $D$ transformation. In the baseline, the $D$
   transformation did not exist, thus this transformation was added to achieve
   field-sensitivity.

   For most IR instructions, explicit flow is constrained by adding the
   constraint $V(source) \rightarrow V(sink)$. There are a subset of LLVM IR
   memory operations that require interaction with \textit{DSA}. The
   store, load and GetElementPtr (GEP) instructions are the most important for
   addressing the issue of field sensitivity. The rules for how these are
   constrained are given in figure \ref{fig:llvm-mem-ir}.

\begin{figure}[h!]
  \hrulefill
  \begin{lstlisting}[style=nolinenumberstyle]
<result> = alloca <type> [, <ty> <NumElements>] [, align <alignment>]
  \end{lstlisting}
  \explicitflow{ $\rightarrow V(\result)$}

  \implicitflow{$V(PC) \rightarrow V(\result)$}

  \begin{lstlisting}[style=nolinenumberstyle]
<pointer> = getelementptr inbounds <ty>, <ty>* <ptrval>{, [inrange] <ty> <idx>}*
  \end{lstlisting}
  \explicitflow{ $V(\ptrval) \rightarrow V(\pointer)$}

  \implicitflow{ $V(PC) \cup V(\ptrval) \rightarrow V(\pointer)$}

  \begin{lstlisting}[style=nolinenumberstyle]
store <ty> <value>, <ty>* <pointer>[, align <alignment>]
  \end{lstlisting}
  \explicitflow{$V(\myvalue) \rightarrow D(\pointer, \offset)$}

  \implicitflow{$V(PC) \cup V(\pointer) \rightarrow D(\pointer, \offset)$}

  \begin{lstlisting}[style=nolinenumberstyle]
<result> = load <ty>, <ty>* <pointer>[, align <alignment>]
  \end{lstlisting}
  \explicitflow{ $V(\pointer) \cup D(\pointer,\offset) \rightarrow V(\result)$}

  \implicitflow{ $V(PC) \cup V(\pointer) \cup D(\pointer,\offset) \rightarrow V(\result)$}
  \hrulefill
  \caption{LLVM Memory IR Flow Rules}
  \label{fig:llvm-mem-ir}
\end{figure}

The load and the store instructions are the instructions which changed the most
from the baseline. In the baseline, these rules were the same, but the offset
was not considered in the $D$ transformation. The store and load instructions do
not have an offset as a operand in their instructions. The offset is found  by
examining the IR instruction which yielded the operand. If the operand is a pointer
computed from a structure or an array, then the pointer will have
been calculated using the GEP instruction. By including the GEP instruction in
the sources for the load (and the sink for the store), the instruction can be
analyzed and the index can be accessed.


\subsubsection{Constraint Generation}

For operations which do not modify memory, the constraints are generated such
that $V(inputs) \rightarrow V(outputs)$. Loads and store instructions require
additional steps to include the points-to analysis. The points-to analysis
represents memory as nodes in a graph. Structures and arrays are represented as
a single node with as many types and offsets as necessary. Figure
\ref{fig:storeconstraint} shows how a store would be constrained using
additional information from the pointer operand and the type information from
the points-to analysis.


\begin{figure}[h!]
\begin{lstlisting}
typedef struct {
  int rd_key;
  int rounds;
} AES_KEY;

AES_KEY pk;
pk.rounds = 3;
\end{lstlisting}

\begin{lstlisting}
  %pk = alloca %struct.AES_KEY, align 4
  %rounds = getelementptr inbounds %struct.AES_KEY, %struct.AES_KEY* %pk, i32 0, i32 1
  store i32 3, i32* %rounds, align 4
\end{lstlisting}
  \caption{Store Constraint Example}
  \label{fig:storeconstraint}
\end{figure}

The $D$ transformation utilizes the points-to analysis to retrieve the correct
memory node. The memory node stores type information for any types accessed from
that node in the IR code. For stack arrays and structures, the type information
from the points-to analysis is used to create individual elements for each of
the members accessible from that structure/array. Thus the offset must be passed
to $D$ along with the pointer to select the correct element. For structures, the
instructions only give an index, so the byte offset is computed using the data
layout information that the points-to analysis uses to create its type information.

In figure \ref{fig:storeconstraint} The \codevar{AES\_KEY} structure has two
fields. The GEP instruction is used to calculate the address of the fields when
operating on the structure. The inputs are the base pointer and zero or more
offsets. The store instruction is constrained using the pointer and the offset
operands, which are provided by the GEP instruction that flows into the store
instruction. In this example, the offset to be used is the last operand. The
last operand is not a byte offset from the base pointer, its an index. Using the
points-to analysis, the index is converted into a byte offset. Bytes are used
instead of the index to account for a types width. The type width matters in the
cases where types are overlapping.

 The rest of this section shows how the analysis handles when the necessary
 information is not available to create a proper field sensitive constraint.

\subsubsection{Factors Affecting Field-sensitivity}
\begin{itemize}
\item{Collapsed Memory Nodes}

   The points-to analysis is type-safe, marking nodes as collapsed when the type
   information is inconsistent. If a node is collapsed, only one constraint
   element is created for that node. Multiple constraint elements for a single
   memory node are only created when the points-to analysis has type information
   available for the node.

\item{Incomplete Type Information}

   The points-to analysis used only provides type information for the values used
   in the analyzed source. The field-sensitive analysis should also be
   conservative when analyzing unknown source. For this, the analysis relies on
   the data layout of structures which is the same information the points-to
   analysis uses to build its type information. The layout information helps in
   the case where a tainted field is not used within the provided source but its
   data structure is passed as an argument to an unknown function. This provided
   the possibility for the tainted field to leak information.

\item{Calls to Unknown Functions}

For functions which the source code is not provided, the constraint rules define
how to treat values and their flow. The rules used, allow for flow between all
pointer types, and also flow of information within the pointer itself. That is,
a structure may only have a single tainted field, but after passing through an
unknown source, all fields are tainted due to possible flow between fields.
Likewise, flow from one pointer to another ensures that any tainted value
propagates in the constraints to other parameters.

\[V_{args} = \bigcup_{i=0}^N V(arg_i)\]
\[R_{args} = \bigcup_{i=0}^{N}\{R(arg_i) : type(arg_i) = pointer\}\]

\noindent
\textbf{Explicit:} $V_{args} \cup R_{args} \rightarrow V(retval) \cup R_{args}$

\noindent
\textbf{Implicit:}$V(PC) \cup V(functionptr) \cup V_{args} \cup R_{args} \rightarrow V(retval) \cup R_{args}$
\end{itemize}

\subsection{Solving Information Flow Constraints}
Given the set of constraints for the program being analyzed, the least solution
where all the sinks are greater than or equal their sources is found, otherwise
known as a least solution. Given a set of initial sensitive values, any sink
which has a value greater than or equal to that value, is considered sensitive.
In the case of cryptosystems, it can be helpful to find all the values which
may have been computed from confidential data. These values are possible
locations for leakage of confidential information, presenting a possible
vulnerability.


Consider figure \ref{alg:simpleflow} for example of how the analysis functions.
First the flow of information must be identified. For simplicity, this example
will just consider the source code, not the instructions. In figure
\ref{alg:simpleflow}, $k$ is just a constant set to some confidential value, no
information flows there. Next, $d.x$ is calculated by adding 1 to the value of
$k$, so there is flow from $k$ to $d.x$. With $d.y$ and $a$, in lines 3 and 4
respectively, they are set with constants, so no flow happens there. Then, a
branch is encountered where the value of $a$ is modified based on the values of
$d.y$ and $k$. Let the branch condition be $b$, there will be flow from $d.y$
and $k$ to $b$. Lastly due to the branch, there is an implicit flow from $d.y$
to $a$ and $k$ to $a$.

For explicit flow the set of constraints is as follows.
\[
  k \leq d.x
\]
\[
  d.y \leq b
\]
\[
  k \leq b
\]
In this case it is easy to see that $d.x$ should be at least as confidential as $k$
since it is directly computed from $k$. Similarly, the branch condition $b$ is
confidential since it must be at least as confidential as $k$. The value $d.y$
is not confidential, but is irrelevant because confidential information is used
to determine the outcome of the branch.

If implicit constraints are to be considered then the set is as follows.
\[
  k \leq d.x
\]
\[
  d.y \leq b
\]
\[
  k \leq b
\]
\[
  b \leq a
\]

In this case the all the constraints are the same apart from the addition of the
last constraint. There is another constraint between the branch condition and a,
because after the branch, the value of a is dependent on the outcome of the
branch. If an attacker had a method of inspecting the value of $a$, they would
learn information about $b$ which is derived from $k$.

The baseline analysis generates a set of constraints similar to the examples
above, but with some imprecision. During the stage of generating constraints
from the information flow, the points-to analysis is leveraged to figure out
what is being pointed to by the variable $d$ to subsequently find the correct
instance of $x$. The points-to analysis will represent the data structure
pointed to by $d$ with one node in the graph, and the constraint generated
refers only to that node, instead of to the specific field in the node. The
effect on the constraints is as follows for explicit flow.

\[
  k \leq d
\]
\[
  d \leq b
\]
\[
  k \leq b
\]

The result is that $d.x$ and $d.y$ are indistinguishable from each other and the
results will report both $d.x$ and $d.y$ as confidential even though manual
scanning of the source shows that $d.y$ is never calculated from confidential
data. This is a common problem when analyzing sources. Take this implementation
of AES, where the data structure used is comprised of both confidential and
public data. The \codevar{AES\_KEY} struct has the confidential \codevar{rd\_key} and a second field
rounds that is considered to be public. Figure \ref{alg:aesstruct} shows
where the inaccuracy can be found in actual source code.

So with the improved analysis, the constraints are generated such that compound
data structures with multiple nodes are able to be constrained separately. Once
the set of constraints for a programs source code is created, the least solution
is found and the results show the line number where the branch occurs. This list
of results shows the branches which are tainted and untrusted.

\begin{figure}[h!]
\begin{lstlisting}
struct AES_KEY_t {
  unsigned * rd_key;
  int rounds;
};

typedef AES_KEY_t AES_KEY;

void AES_enc(char * in, char * out, AES_KEY) {
  unsigned * rk = key->rd_key;
  if (key->rounds > 10) {
    ...;
  }
}
\end{lstlisting}
\caption{Public and private data in structure}
\label{alg:aesstruct}
\end{figure}

\subsection{Result Classification}

The analysis provides a list of source code lines which are vulnerable. Based on
the contents of the input files, a branch is flagged as vulnerable if the
condition depends on data which has been marked as tainted or untrusted. The
reported lines can then be reviewed and sorted to rank them in order of
severity. Classification of the vulnerable branches consists of three stages,
the first removes error-handling results, the second sorts the remaining results
into either a high-risk or low-leakage sets, and the final sorts the high-risk results
depending on the surrounding source context.

Stage 1 is a filter which removes the error handling and input validation
results. For normal operation of a program, these branches check sensitive data
but allow an early exit if the branch condition evaluates to true. In figure
\ref{alg:stage0src}, the variable \texttt{N} is tainted, so the branch is
vulnerable due to the condition depending on \texttt{N} and the data held within
\texttt{N}. The result of this branch is not helpful in the case of an attacker
attempting to learn the value of \texttt{N}. As long as the input is valid, the
condition of this branch will always evaluate to false. If the branch leads to
exit code, it is a candidate for removal in this stage. To conclude stage 1, if
there are results which are reported that leak non-sensitive data, they may be
whitelisted. The whitelist allows results to be ignored, and not propagate taint
as a result in the later stages.

\begin{figure}[h!]
\begin{lstlisting}
if( mbedtls_mpi_cmp_int( N, 0 ) <= 0 || ( N->p[0] & 1 ) == 0 )
    return( MBEDTLS_ERR_MPI_BAD_INPUT_DATA );
\end{lstlisting}
\caption{Error-handling Source Code}
\label{alg:stage0src}
\end{figure}

Stage 2 operates on the remaining results to sort them into high and low-risk
categories. A high-risk branch will be due to one of the operands being either
directly related to the sensitive data or derived from more than 1 bit of
sensitive data. Low-risk branches, are branches which the operand is based on 1
bit of the key or the same operand could derived from some set of sensitive data
values.

\begin{figure}
\begin{lstlisting}
#define MPN_NORMALIZE(d, n) \
do {                        \
  while( (n) > 0 ) {        \
    if( (d)[(n)-1] )        \
      break;                \
    (n)--;                  \
  }                         \
} while(0)

MPN_NORMALIZE(ep, esize);
if (esize * BITS_PER_MPI_LIMB > 512)
  W = 5;
\end{lstlisting}
\caption{High-Risk and Low-Risk Branches}
\label{alg:branchriskexample}
\end{figure}


Figure \ref{alg:branchriskexample} is a snippet of some of the modular
exponentiation code from Libgcrypt 1.8.2. For instance, say the values pointed
to by \texttt{ep} are marked as tainted at the start of the analysis, and the
analysis reports that lines 10 and 11 are vulnerable. Line 10 is the macro
defined in the lines 1-8. Line 10 is reported due to the macro having the
\texttt{if} branch directly dependent on the data pointed to by \texttt{ep}.
Since the branch is directly dependent on the tainted data that branch is
considered high-risk. Line 11 however is a branch based on \texttt{esize}, the
value begin modified within the \texttt{MPN\_NORMALIZE} macro. The value of
\texttt{esize} is not unique to the tainted data, meaning there exists a set of
tainted values which may yield the same \texttt{esize}. If a value is derived
from tainted data, as \texttt{esize} is, but the result is not unique to that
instance of tainted data, it is considered low-risk.

If for any reason, the classification is unclear, such as a mutable type being
passed to source code that is not analyzed, the data is treated as high risk.
This is done in order to be conservative. Further categorization can be
performed on the high-risk set of results in Stage 3. Additionally, if there are
a large number of low-risk result, Stage 2 can be repeated, by appending the
low-risk variables to the whitelist.

Stage 3 specifically looks a the context of the high-risk results from Stage 2.
Each of the results will be classified as one of the following: a single branch, repeated
branch, controllable branch or both (e.g. both controllable and repeated).
Single branches are branches which are not within a loop. Repeated branches are
branches that are part of a loop definition or within the body of a loop.
Controllable branches are that which the branch result is both untrusted and
tainted. The purpose of sorting results this way is to be able to prioritize
which vulnerable branches to examine first when attempting to make the
application secure.

\section{Evaluation}
\subsection{Implementation}

Achieving the improved precision of the analysis is done primarily by
handling LLVM's pointer instruction differently than other instructions. The
pointer instruction in the IR is the GEP instruction. This instruction
is used when computing an address from a base pointer. For arrays, for example
it has the index of the element and the bounds of the array if they are known.
For structures, the instruction contains the structure type which is being
addressed and the index of the field in the structure that is being referenced.
The baseline analysis lacked the ability to consider the index available in
these instructions. The points-to analysis had type and offset
information for the targets of the pointers. Improving the analysis was achieved
by i) creating the appropriate number of constraint elements for each node in
the points-to graph, ii) converting the field index from the GEP instruction to
a byte offset, and iii) constraining the correct elements based on
the GEP instruction and data type information.

\subsubsection{Constraint Element Generation}
When a pointer value is encountered for the first time, a node is created for it
in the points-to graph and a set of constraint elements are created for that
data structure. The points-to analysis was left unmodified, so the focus will be
on the semantics of how constraint elements were changed. In the baseline, only
one constraint element represented any target of a pointer. In the improved
analysis, type information from the points-to analysis was used to generate the
appropriate number of constraint elements. In a stack allocated array, one
constraint element was created for each element in the array. For structures and
classes, each field is located at some offset away from the base address of the
structure and a constraint element was created for each field and mapped to
associated byte offset. This change enables the precision of the baseline to be
improved by correctly selecting the corresponding constraint element based on
the IR.

For any GEP instruction, constraint elements are generated using the type
information from the points-to analysis when available. Each constraint element
is created by iterating through the type information of the node from the
points-to analysis. For each type there is an associated byte offset, using this
offset and the size as reported by LLVM, a constraint element is created with
the corresponding starting byte offset and the ending byte offset based on the
width of the field. It is necessary to account for the width and start location
of each field because there may be padding between fields of a structure.
Padding between fields may be present in the type information because of an
unused field in code. The points-to analysis will not have type information on
unused fields so it is important to account for these gaps. Another reason to
account for the padding is due to structures which are not aligned with each
field following sequentially after the previous field. When the type information
is not available one constraint element is created for the whole node so at
worst the improved analysis will be equivalent to that of the baseline.

Figure \ref{alg:generateconselem} shows how the points-to analysis is
leveraged to create constraint elements for the proper byte offsets and widths.
Let \texttt{s} be the LLVM representation of the structure type referenced by
the GEP instruction, and \texttt{node} be the node in the memory graph provided
by the points-to analysis. The algorithm works by retrieving the graph in which
the node resides and then retrieving the data layout as specified by the
compiler for that graph. The data layout contains information like the field
lengths of each type and the alignment within a data structure. The structure
type itself \texttt{s} does not have the alignment and padding information, only
the types of each field within the data structure. The data layout is used to
retrieve the structure layout of the structure type from the GEP instruction.
The structure layout is vital because it allows the index from the GEP
instruction to be converted to a byte offset within the data structure. The
data layout also provides the size of each type, so that the constraint element
is created at the corrects starting and ending offset.


\begin{figure}[h!]
  \lstinputlisting{examples/createConsElem.cpp}
  \caption{Creating constraint elements for each field in a type}
  \label{alg:generateconselem}
\end{figure}

The elements created using this strategy enable the analysis to be more precise.
GEP instructions can also be used to index arrays, so that is handled by
computing the number of elements that array may hold. For stack arrays that
number is known, but for heap arrays the number is variable. For stack arrays
each, since the number of elements is known, one constraint element is created
per array element. For heap arrays, one constraint element is created for the
entire array.

Similarly, if type information is unavailable then a conservative approach is
used. One constraint element is created for the entire data structure. When
the information is available, the analysis is able to be improved by
constraining the correct constraint element for each instruction. When that
information is unavailable, the improved analysis is unable to be more precise
than the baseline analysis.

\subsubsection{Constraining Operands}
Each time an instruction where information flow exists, constraints are
generated between the operands of that instruction. The idea of this is simple,
given a set of constraint elements select the ones which are used in the
operation and constrain them as necessary. The baseline analysis handled
individual variables in a precise manner as long as the variable was not a
structure or array. The baseline analysis had no way of identifying the offset
from a GEP instruction, and even if it did, the mapping between a memory node to
a constraint element was one-to-one. In the previous section, the mapping from
memory node to constraint element was made to be one-to-many. As one node may
represent multiple fields which together make the structure or array. It is now
necessary to pick the correct element from this one-to-many mapping when
generating the constraint for each instruction. Again, this is done by analyzing
the GEP instruction.

The last operand of a GEP instruction is the index of the field within the data
structure. Each constraint element for a memory node is created for a byte
offset range, so it is necessary to correctly calculate the byte offset from the
field index provided by the GEP instruction. This is very similar to the
strategy used to generate the constraints at the correct byte locations. Given
the structure type, field index and node from the points-to analysis, the offset
is calculated as shown in figure \ref{alg:findoffset}. Given the node, the
data layout and associated structure layout can be found and then the index can
be converted to a byte offset.

\begin{figure}[h!]
\lstinputlisting{examples/findoffset.cpp}
\caption{Calculating byte offset from index}
\label{alg:findoffset}
\end{figure}

If the GEP instruction is in reference to an array access, then the offset is
just calculated using the size of the element type and the index. If no type
information exists for the node, only a single constraint element would exist
for the node, so that is the constraint which is selected to be constrained.

\subsubsection{Missing Type Information}

The type information used to generate the constraints came from the points-to
analysis. The analysis is conservative in the type information it
provides. This means that the type information given is limited to only fields
that have been used in the source, and the type information is cleared if the
analysis can not make definitively determine the types along with their offsets.
The requirement to have a field used in the source means that often the type
information from the structure is incomplete due to unused fields or fields
which external functions may use. The points-to analysis clearing type
information removes the ability to have a field sensitive constraint at all.

In the case of missing the unused data information, the solution is to utilize
the information that the points-to analysis uses to build the type map. Given a
structure type in LLVM, the list of fields and their types are known. The
points-to analysis relies on the data layout provided in the bitcode. Constraint
elements are created by querying the data layout to identify the initial byte at
which the type starts and the width of the type to find where that type should
end. The start and end of the elements are recorded when creating the constraint
elements for that type.

Building the type information always instead of on-demand is necessary when
there are external function calls where the source is not linked. If a pointer
which holds sensitive data is a parameter for that function, then it is possible
that sensitive data may flow to other arguments or the return value. When the
field-sensitivity was implemented, some of the higher-risk results had been
removed. This happened because the pointer to sensitive data was no longer
considered sensitive as it was when offset was unused. Before, it was enough to
see if any of the arguments were sensitive, and create flows from the sensitive
data to all sinks from that function. After the field sensitivity change, the
pointer and its data are constrained separately, so when external calls happen,
the analysis creates constraints from each of the fields which are reachable
from that pointer in addition to the normal flows which are provided by the
function call.

If there is no type information at all, then the improved and baseline analysis
function the same way, there is one element that represents that type and that
one element is constrained regardless of offset. The points-to analysis denotes
unsafe type information by collapsing the node in the memory graph.
\subsection{Benchmarks}

The baseline analysis had many false positives and low-risk positives. Improving
the analysis would mean reducing the positives to only results which are
possible flaws. The benchmark results shown in table
\ref{tbl:overall-feature-benchmark} show the effects of different features on
the number of reported secret-dependent branch warnings. There are 2 features
implemented which were looked at in isolation, the white-list and the
field-sensitivity changes. Additionally, adding sources for functions which are
not defined in the file being analyzed allows for a better understanding of
information flow.

The baseline benchmark evaluated the modular exponentiation code for each
library. This benchmark was run without white-list or field sensitivity. The
source provided was that only which was necessary to compile the modular
exponentiation code. All multi-precision integer (mpi) arithmetic library components
were omitted.

The field-sensitive (FS) test is the same as the baseline benchmark but with
field-sensitivity enabled. The results from this test differ depending on the
library. For example, OpenSSL did not see any benefit from FS alone. This is due
to openSSL doing much of the math in external functions, and each external
function return value was checked in the modular exponentiation code. Since
field-sensitivity requires source to work, it makes sense that openSSL did not
see benefit in this test. The reasoning is the same for the result in mbedTLS.
Libgcrypt however does manipulate the tainted data within the source analyzed so
it did benefit from FS over the baseline benchmark.

The white-list (WL) test is the same as the baseline benchmark but only using a
white-list to remove results which are clearly false or low-risk positives from
the baselines results. The white-list allowed the libraries which did not
benefit from field-sensitivity to see a reduction in positives. Libgcrypt saw an
improvement as well but not nearly as much as the field-sensitive change.

The full source (SRC) test is the same as the baseline but with additional
source files provided. The additional sources provided were the remainder of the
multi-precision integer or big number(bn) libraries. This meant that all mpi or bn math
library calls were known functions. The important fact here is that the flows
were more direct now between parameters in functions. Without field-sensitivity
though all data-structures and arrays are treated as one element. This improved
the number of results in all the libraries which did not benefit from
field-sensitivity.

The last two tests were combining features to see if greater improvement was
possible. The white-list and field-sensitivity test was chosen since the
field-sensitivity is not able to function when external functions are called
frequently, but white-listing can be used to compensate. The last test combined
the 3 individual tests, using the baseline analysis with field-sensitivity, a
white-list and the entire mpi/bn libraries.

The full combination of improvements over the baseline is where we get to see
that adding additional source in combination with the field-sensitive changes
improved the results in OpenSSL and mbedTLS the most. Libgcrypt however did see
an increase in the number of positives, which is result based on the limitations
of the points-to analysis used. The points-to analysis, makes use of
heap-cloning to track distinct nodes which may be processed by a common source
function. The analysis also merges nodes that are found to be in the same
equivalence class. In the case of Libgcrypt, nodes that were considered distinct
in the previous benchmarks, end up aliased to the same node in the full
benchmark. This causes an inferior result to the WL/FS test.

\begin{table*}
  \centering
  %\begin{tabular}{|p{3.5cm}|p{1.5cm}|p{1.5cm}|p{1.5cm}|p{1.5cm}|p{1.5cm}|p{2.5cm}| p{2.5cm}|}
    \begin{tabular}{@{}lrrrrrrr@{}}
    \toprule
    Library & Base & FS & WL & SRC & WL/FS & WL/FS/SRC & \% Reduction\\
    \midrule
    \textbf{Libgcrypt 1.8.2} & 64 & 32 & 38 & 64 & 18 & 24 & 62.5\%\\
    \textbf{mbedTLS 2.9.0} & 40 &40 & 35 & 25 & 35 & 22 & 45.0\% \\
    \textbf{BearSSL 0.5} & 1 & 1 & 1 & 1 & 1 & 1  & 0\%\\
    \textbf{OpenSSL 1.1.0g}\\
    \hspace{0.25cm}Reciprocal             & 32 & 32  & 20    & 24   & 20 & 13  & 59.4\%\\
    \hspace{0.25cm}Mont.            & 38 & 38  & 28    & 30   & 28 & 21  & 44.7\%\\
    \hspace{0.25cm}Mont Const. Time & 30 & 30  & 27    & 16   & 27 & 14  & 60.0\% \\
    \hspace{0.25cm}Mont. Word        & 31 & 31  & 21    & 21   & 21 & 12 & 54.8\% \\
    \bottomrule
  \end{tabular}
  \caption{Number of Warnings based on Features}
  \label{tbl:overall-feature-benchmark}
\end{table*}

\subsection{Case Study}
%The case study serves to demonstrate the changes to the results with the
%improved analysis over the baseline. Each version of the analysis is compared by
%analyzing the modular exponentiation code used in SSL and TLS libraries.
%OpenSSL and Libgcrypt were selected due to vulnerabilities that were able to be
%found by this analysis. The mbed TLS uses the same modular exponentiation as
%OpenSSL and libgcrypt but does not provide constant time implentations. BearSSL
%is selected due to its claim to be a constant time library.

The baseline and improved analysis were compared using current software TLS and
SSL libraries. The libraries analyzed were Libgcrypt, OpenSSL, mbed TLS, and
BearSSL. The case study was done by comparing the modular exponentiation
algorithms between each of the libraries. The Libgcrypt and OpenSSL libraries
were chosen due to their wide-spread use. The mbed TLS library is built for
embedded platforms, so this was chosen to search for shortcuts that may provide
vulnerabilities. BearSSL is chosen because it claims to be a constant-time
cryptographic library.

\subsubsection{Experiments}
The parameters used to test the libraries were set such that the data alone was
set to be tainted, meaning that no other members which shared the same structure
should be treated as sensitive data. The results of the improved analysis should
eliminate many of the false-positives, and some of the low-risk results. The
number of high-risk results should not change.

The source code which implemented modular exponentiation functions were analyzed in the
following configurations:
\begin{enumerate}
\item Baseline Analysis
\item Baseline with field-sensitivity changes only
\item Baseline with white-list only
\item Baseline with full library source code
\item Baseline with whitelist and field-sensitivity
\item Baseline with field-sensitivity, whitelist and full library source code
low-risk results.
\end{enumerate}

%% \begin{table}
  %% \caption{Baseline and Improved Results - Same Input}
  %% \begin{tabular}{|l|r|r|r|r|r|r|r|r|r|r|}
    %% \hline
%% Library & Num. Branches & Baseline & Positives & False Positives & Low-Risk & Validation & Single & Looped & Controlled & Both \\
    %% \hline
%% LibGcrypt 1.8.2 &	246 &	69 &	68 &	2 &	47 &	3 &	4 &	12 &	0 &	0 \\
%% mbedTLS 2.9.0 &	1471 &	25 &	25 &	0 &	18 &	3 &	0 &	3 &	1 &	0 \\
%% BearSSL 0.5 &	141 &	36 &	10 &	0 &	10 &	0 &	0 &	0 &	0 &	0 \\
    %% \hline
%% OpenSSL 1.1.0g &  & & & & & & & & & \\
%% recp	& 498 &	1 &	24 &	0 &	11 &	11 &	0 &	2 &	0 &	0 \\
%% mont	& 498 &	6 &	34 &	0 &	14 &	16 &	0 &	2 &	2 &	0 \\
%% mont\_consttime	& 498 &	8 &	30 &	0 &	16 &	12 &	0 &	0 &	2 &	0 \\
%% mont\_word	& 498 &	11 &	22 &	0 &	5 &	13 &	1 &	1 &	0 &	2 \\
%% simple	& 498 &	0 &	22 &	0 &	12 &	8 &	0 &	2 &	0 &	0 \\
    %% \hline
  %% \end{tabular}
  %% \label{tbl:baseline_improved_same}
%% \end{table}


\subsubsection{Potential Sources of Unsoundness}
For these experiments, the results given may not be all the results possible
from the source analyzed due to underlying assumptions. Since the full source
code was not analyzed, there are function calls which are defined in files which
are not compiled and linked. One assumption made for unknown functions is that
the flow only exists between the parameters. There is a possibility for this
assumption to break down if, for example, pointers are modified to point to
something which is not a value provided in the parameter. Another potential
cause of missing results is due to only considering explicit flows. The analysis
is able to handle both explicit and implicit flow, but currently they are only
able to be run independently. The experimental results shown only consider
explicit flows.

\subsubsection{Performance}

\begin{table*}[!t]
  \centering
    \ra{1.2}
  \begin{tabular}{@{}lcrrrrcrrrr@{}}
    \toprule
     & & \multicolumn{4}{c}{Minimal Source} & &  \multicolumn{4}{c}{Full Source} \\
    \cmidrule{3-6} \cmidrule{8-11}
    Library&& Baseline & Improved & \# Branches & KSLOC && Baseline & Improved & \# Branches & KSLOC\\
    \midrule
    Libgcrypt 1.8.2 && 00:24 & 00:24 & 246 & 6.5 &&  17:42 & 19:37 & 1938 & 11.1\\
    BearSSL 0.5 && 00:21 & 00:21 & 141  & 3.6 && 00:21 & 00:27  & 141 & 3.6\\
    mbedTLS 2.9.0 && 00:07 & 00:07& 214 & 0.5 && 01:14 & 01:45  & 1471 & 1.7\\
    OpenSSL 1.1.0g && 00:27 & 00:29& 498 & 18.7 && 11:00 & 13:11 & 1880 & 28.8\\
    \bottomrule
  \end{tabular}
\caption{Baseline and Improved Analysis Run Time (mm:ss)}
\label{tbl:runtimes}
\end{table*}

The performance running times of the baseline is compared with the best
performing improved analysis in Table \ref{tbl:runtimes}. For each library, the
test was done with both the full source (including the full big number library)
and the minimal source (only the modular exponentiation code). The
tests show how difference in size of the program affects the running time. The comparison
also shows that the increase in processing time in the improved analysis is
small even for the larger libraries such as libgcrypt and OpenSSL.

\subsubsection{Result Classification}
 Of all the experiments, two of the tests were classified to understand the
 category for the positives. The first experiment classified is the baseline
 experiment. Ideally no high-risk results are eliminated when moving from the
 baseline to a higher precision result. For most of the libraries the best
 results were given when combining the full source with the analysis that
 utilized field-sensitivity and white-listing, so those results were classified
 to compare against the baseline.

 The results from the experiments are sorted into four categories:
 false-positives, error-handling, low-risk and high-risk. The categories are used to
 build a white-list to further refine the results from the analysis.

 Results which are classified as false positives are marked due to not being
 calculated from sensitive data, being marked sensitive as a result of
 imprecision. This analysis is not flow-sensitive, so results which involve
 values which are re-computed from sensitive data later in an execution are also
 considered false-positives. In the baseline analysis, false positives show up
 due to the fact that a structure or an array is treated as one entity. Fields
 within that structure or array which have not been computed from sensitive data
 are reported falsely.

 Error-handling results are branches which lead to exit or error handling code.
 Error-handling branches will not execute for valid inputs. For any valid input
 then, the result of this branch will be the same. The branch may check
 sensitive data but if the result of the branch is known to an adversary, only
 trivial information would be gained (i.e. whether or not the data is valid).

 Results which are not in the first two categories are then sorted between the
 low- and high-risk sets of results. The distinction between a low and a
 high-risk result is on how much information is contained about sensitive data
 if the outcome of a branch result is known. High-risk means that there is a
 possibility to leak multiple bits of sensitive data from either a single or
 repeated execution of the branch. Low-risk results can be identified if the
 values in the branch condition can be the same across a set of sensitive
 values. For example, many number libraries have a branch based on the length of
 a sensitive value, but that value is going to be the same for a range of
 values.

 If the analysis reports a large number of results, it can be helpful to utilize
 the classifications done to remove entries from the list. The white-list can be
 compiled by identifying the variables which propagate sensitive data but is
 considered acceptable. The white-list is often very small yet can eliminate
 many of the false-positive or low-risk results.

 The second stage of classification was done after adding a small set of
 variables to the white-list. This stage breaks down the high-risk category from the
 previous stage based on the surrounding context. The first criterion is whether
 or not the branch is within a loop. The second criterion is if the branch is
 controllable. The high-risk results can then be classified as either, single,
 repeated, controllable, or loop-controllable.

 Controllable means that the branch outcome is dependent on input from an
 untrusted source. In the case of modular exponentiation, the plain-text used in
 the cryptographic algorithms is untrusted, ignoring the effects of blinding. So
 a branch would be controllable if there was a comparison between the untrusted
 data and tainted data such as the power or modulus.

 Single branches are branches are not within a loop and are not controllable.
 Controllable branches are not within a loop. Repeated branches are within a
 loop and are not controllable. Loop-controllable branches are branches which
 are within a loop and are controllable.

\subsubsection{Experiment Results}

    In this section results are shown from the library test which had the lowest
    number of positives while not losing high-risk results. Ideally, the total
    number of positives decreases and the high-risk results are not sacrificed
    for that outcome. The looped and controllable results are the ones that are
    considered high-risk. The summary for all the libraries is given in Table
    \ref{tbl:allpositives}. After each libraries results are discussed to
    understand both the false positives and the high-risk results.

\begin{table*}[!t]
  \centering
  \begin{tabular}{@{}lrr|crrrcrr@{}}
    \toprule
    &  &  &&  & \multicolumn{2}{c}{Low-Risk} && \multicolumn{2}{c}{High-Risk}\\
    \cmidrule{6-7} \cmidrule{9-10}
    Library& Base & Improved&& FP& Error-Handling & Low Leakage && Looped & Controllable \\
    \midrule
    \textbf{Libgcrypt 1.8.2}& 64 & 24 &&8 & 10 & 1 && 5 & 0\\
    \textbf{mbedTLS 2.9.0}  & 40 & 25 &&2 & 13 & 3 && 3  & 1\\
    \textbf{BearSSL 0.5}    & 1 & 1   &&0 & 1 & 0  && 0  & 0\\
    \textbf{OpenSSL 1.1.0g}\\
    \hspace{0.25cm}Reciprocal              & 32 & 13 && 3    & 2   & 6 && 2 & 0\\
    \hspace{0.25cm}Montgomery              & 34 & 21 &&  3   &  5  & 9 && 2  & 2\\
    \hspace{0.25cm}Montgomery Const. Time  & 30 & 12 &&  0   &  7  & 3 && 0   & 2\\
    \hspace{0.25cm}Montgomery Word         & 22 & 14 &&  2   & 6   & 5 && 1   & 0\\
    \bottomrule
  \end{tabular}
\caption{Result Classifications: Base - Baseline Positives, Improved - WL/FS/SRC
Positives}
  \label{tbl:allpositives}
\end{table*}


All libraries (BearSSL excluded) saw an improvement from adding
field-sensitivity, whitelist and/or additional source. The configuration which
yielded the best results was including all 3 options. BearSSL did not improve
with all 3 configuration options, but all the other libraries saw a reduction in
low-risk or false positives. Libgcrypts best configuration did not include
additional source code. Adding extra source code to the tests did not yield the
least positives out of all the benchmarks for that library.


Libgcrypt had many false-positives removed between the improved and the baseline
analysis. The false positives here were due to the multi-precision integer
structures having multiple fields containing public data. Many of the branches
flagged as vulnerable were falsely reported due to the baseline analysis not being field-sensitive.

 In the mbed TLS library had no changes when the offset was used, so the results
 here were not due to field-sensitivity issues. BearSSL does not use a structure,
 but instead just an array to represent multi-precision integers. There were no
 additional fields so field-sensitivity had no effect on the results for this
 library either. The false positives however were due to the length being
 computed from the data within the integers.

 OpenSSL showed some reduction in the number of results while maintaining all of
 the high-risk results. The number of false positives did not change, but the
 number of low-risk results were removed by adding field-sensitivity. OpenSSL
 does not modify the multi-precision integers in the files analyzed so many
 calculations and the flow between them are uncertain.

\subsubsection{Libgcrypt High-risk results}


     Libgcrypt has a multi-precision integer library with a file specifically
     for modular exponentiation. This library operates on the inputs within this
     source file, only executing external function calls for other math
     operations. Unlike mbedTLS and openSSL, the branches are based on the
     inputs instead of the return value of external functions. This is evident
     by the fact that adding field-sensitivity alone reduced the number of
     positives from the baseline. The best result for this library was when
     white-listing and field-sensitivity were used together.

     The first high-risk results comes from the MPN\_NORMALIZE call, which is a
     macro, which scans the exponent pointer and sets the size accordingly. This
     is done in a branch while directly checking the value of the sensitive
     data. Once the first non-zero element is found, esize reflects the position
     of that element. The source is shown in figure \ref{code:libgcrypt_normalize}.

\begin{figure}[h!]
\begin{lstlisting}
#define MPN_NORMALIZE(d, n)    \
    do {		                   \
      while( (n) > 0 ) {       \
        <@\textcolor{red}{if( (d)[(n)-1] )}@>          \
            break;	           \
        (n)--;	               \
      }		                     \
} while(0)
\end{lstlisting}
\caption{Libgcrypt 1.8.2 - mpi-internal.h lines 113-120}
\label{code:libgcrypt_normalize}
\end{figure}

This code is listed within the macro \texttt{MPN\_NORMALIZE} and the input
parameters are a pointer $d$ and a size $n$. The exponent data pointer is passed
as $d$ and its corresponding size $n$. The if branch is then dependent on the
sensitive data stored at the pointer. This result is interesting because it is
not a constant-time implementation.

The next four results are within the main processing loop portion of the modular
exponentiation code. The main loop of the has a precomputed set of powers and
the correct member of that set is selected based on the set bit of the exponent.
The first high-risk result from this comparison is shown in line 2 of the
abbreviated main loop code shown in figure \ref{code:libgcrypt_mainloop}.

The multi-precision integer structure represent large numbers by partitioning
each number into multiple \textit{limbs}. Each of these limbs is
processed individually in this loop. For each iteration of the loop $e$ is set
to the current limb being processed. The \textit{count\_leading\_zeros}
is called to ensure the first bit in the limb is a set bit. Along with the $e$
there is a variable \codevar{c} which tracks how many bits are to be processed in the
limb. The value of \codevar{c} changes between each iteration and is dependent on the
value of the exponent. Since, \codevar{e} is the data of one limb of the exponent
directly, and \codevar{c} is computed directly from the limbs data, these are high-risk
variables.

Within the main processing loop, the first branch is on $e$ checking if there
are no set bits in the limb $e$ (line 3, figure \ref{code:libgcrypt_mainloop}).
If this is the case, the process just moves on to the next limb. If there are
set bits, in $e$ the rest of the limb is scanned in order to compute the result.
If the outcome of this branch is determined, then the value of the entire limb
is known and this is a high-risk result.
\begin{figure}[h!]
\begin{lstlisting}
e = ep[i];
e = (e << c) << 1;
...
for(;;)
  <@\textcolor{red}{if (e == 0)}@>
    {
      j += c;
      if ( --i < 0 )
        break;

      e = ep[i];
      c = BITS_PER_MPI_LIMB;
    } else ...
\end{lstlisting}
\caption{Libgcrypt lines 609-626}
\label{code:libgcrypt_mainloop}
\end{figure}


The operations done within the else are done to select the proper precomputed
values based on the value of the limb. The variable $c$ is the position of the
first set bit within a limb.  The value of $c$ is computed by looking
at each bit from a limb in the exponent, making $c$ high-risk. The branch on
line 6 (figure \ref{code:libgcrypt_c_window}) is in the main processing loop, so it is classified as a looped high-risk result.


\begin{figure}[h!]
\begin{lstlisting}
count_leading_zeros (c0, e);
e = (e << c0);
c -= c0;
j += c0;

e0 = (e >> (BITS_PER_MPI_LIMB - W));
<@\textcolor{red}{if (c >= W)}@>
  c0 = 0;
else
  {
    if ( --i < 0 ) {
        e0 = (e >> (BITS_PER_MPI_LIMB - c));
        j += c - W;
        goto last_step;
      }
    else
      {
        c0 = c;
        e = ep[i];
        c = BITS_PER_MPI_LIMB;
        e0 |= (e >> (BITS_PER_MPI_LIMB - (W - c0)));
      }
  }
\end{lstlisting}
\caption{Libgcrypt lines 635-658}
\label{code:libgcrypt_c_window}
\end{figure}

At the end of each loop iteration, a non-constant time for-loop runs (Fig \ref{code:libgcrypt_non_const_for}). The
for-loop varies with the number of leading zeros for the exponent. The value of
$j$ depends on the \codevar{c0}, a value derived from the limb of the exponent. This
makes \codevar{j} a high-risk variable, and the for loop branch iterates $j > 0$ meaning
that the number of iterations of the loop is not constant time. In this section
of the source, there are precautions taken to target some timing channels such
as indexing into the precomp array. The conditional set makes it such that for
each value of \codevar{k}, the array value is stored. Similarly, to mask the size of the
answer stored in \codevar{base\_u}, the \codevar{base\_u\_size} is computed and set each time, but only
the bit mask allows only the true size to be set.

\begin{figure}[htpb]
\begin{lstlisting}
  <@\textcolor{red}{for (j += W - c0; j >= 0; j--)}@>
    {

      /*
        *  base_u <= precomp[e0]
        *  base_u_size <= precomp_size[e0]
        */
      base_u_size = 0;
      for (k = 0; k < (1<< (W - 1)); k++)
        {
          w.alloced = w.nlimbs = precomp_size[k];
          u.alloced = u.nlimbs = precomp_size[k];
          u.d = precomp[k];

          mpi_set_cond (&w, &u, k == e0);
          base_u_size |= ( precomp_size[k] & (0UL - (k == e0)) );
        }

      w.alloced = w.nlimbs = rsize;
      u.alloced = u.nlimbs = rsize;
      u.d = rp;
      mpi_set_cond (&w, &u, j != 0);
      base_u_size ^= ((base_u_size ^ rsize)  & (0UL - (j != 0)));

      mul_mod (xp, &xsize, rp, rsize, base_u, base_u_size,
                mp, msize, &karactx);
      tp = rp; rp = xp; xp = tp;
      rsize = xsize;
    }
\end{lstlisting}
\caption{Libgcrypt mpi-pow.c lines 667-695}
\label{code:libgcrypt_non_const_for}
\end{figure}

The while loop shown in figure \ref{code:libgcrypt_whilej} is done after the
main processing loop. The value of \codevar{j} is related to the number of
non-set bits in the last processed limb making it a high-risk variable. The
branch is the condition for a loop, so this is a looped high-risk result. This
result was covered in an attack paper, which showed that knowing the sequence of
squares and multiplies could lead to leakages of set bits in the exponent. This
attack could function by having a program run on the same machine that is
executing the vulnerable code\cite{bernstein2017sliding}. Another attack is based on this same code, which
does not require a program running on the machine, it uses a hardware technique
to measure electro-magnetic emanations from laptops\cite{genkin2015stealing}.

Results like these are important because the attacks are sophisticated, but are
based on the knowledge that information about the target information is
available in the sense to be measured. Gaining knowledge that this branch is
possibly tainted by confidential data does not guarantee a side-channel. This
result is helpful to identify code where a side-channel may exist.

\begin{figure}[htpb]
\begin{lstlisting}
<@\textcolor{red}{while (j--)}@>
  {
    mul_mod (xp, &xsize, rp, rsize, rp, rsize,
                          mp, msize, &karactx);
    tp = rp; rp = xp; xp = tp;
    rsize = xsize;
  }
\end{lstlisting}
\caption{Libgcrypt mpi-pow.c lines 702-707}
\label{code:libgcrypt_whilej}
\end{figure}


\subsubsection{mbedTLS 2.9.0}

     The big number library for mbedTLS is a single source file. For the
     baseline results, the file was modified to only include the source required
     to compile and analyze the modular exponentiation code. The least amount of
     positives was found when combining, field-sensitivity, white listing and
     including the entire big number source. The high risk results in this
     library are branches which look at a single bit in the exponent.

The mbedTLS library had no change in the number of positives with the
field-sensitivity alone. However combining field-sensitivity with the full
library source code improved results.

In the modular exponentiation code, there is a reduction of the base \codevar{a}
if it is greater than the modulus \codevar{n}. The modulus \codevar{n} is marked to be tainted
and \codevar{a} is the multi-precision integer that represents the base. The base is the
user input, and is untrusted. This branch on line 1673 of the source file (\codefile{bignum.c})
directly compares the value of \codevar{a} to the value of \codevar{n}. Although
the modulus is a public value, this result shows that branch conditions that
lead to paths which take time differences based on execution path. Here there is
an untrusted input, the plain-text represented as an integer \codevar{A}
compared directly with a value marked sensitive \codevar{N}. Depending on the
value of the untrusted data in regards to the sensitive data, the code either
does a copy operation or conducts a modulus operation.

\begin{figure}[h!]
\ruleabove
\begin{lstlisting}
<@\textcolor{red}{if( mbedtls\_mpi\_cmp\_mpi( A, N ) >= 0 )}@>
    MBEDTLS_MPI_CHK( mbedtls_mpi_mod_mpi( &W[1], A, N ) );
else
    MBEDTLS_MPI_CHK( mbedtls_mpi_copy( &W[1], A ) );
\end{lstlisting}
\rulebelow

\caption{mbedTLS 2.9.0 - bignum.c lines 1672-1676}
\end{figure}

The 3 looped results reside in the loop which inspects the bits of the exponent.
The data for the limbs containing the exponent is located at \codevar{E$\rightarrow$p}. Each
iteration through the loop looks at one bit of the exponent directly, making
\codevar{ei} a high-risk variable in this loop. The first branch on \codevar{ei} on line 1736 is
a high-risk looped branch to skip leading zeros, so that each window starts on a
set bit. The second branch on line 1739 runs after a full window has been
processed and the subsequent bits are zero. This is again a similar
implementation of the sliding-window exponentiation in Libgcrypt. Based on the
value of the private key exponent, a certain sequence of square operations
happens and once a complete window has been processed, the multiply operation
occurs. This was resolved differently than libgcrypt, mbedTLS used additional
montgomery reductions alongside data blinding\cite{schindler2000timing}, but even that was later
discovered to be vulnerable\cite{walter2001distinguishing}.

\begin{figure}[h!]
\ruleabove
\begin{lstlisting}
while( 1 )
{
   ...
    ei = (E->p[nblimbs] >> bufsize) & 1;

    /*
      * skip leading 0s
      */
    <@\textcolor{red}{if( ei == 0 \&\& state == 0 )}@>
        continue;

    <@\textcolor{red}{if( ei == 0 \&\& state == 1 )}@>
    {
        /*
          * out of window, square X
          */
        MBEDTLS_MPI_CHK( mpi_montmul( X, X, N, mm, &T ) );
        continue;
    }

    /*
      * add ei to current window
      */
    state = 2;

    nbits++;
    wbits |= ( ei << ( wsize - nbits ) );

    if( nbits == wsize )
    {
        for( i = 0; i < wsize; i++ )
            MBEDTLS_MPI_CHK( mpi_montmul( X, X, N, mm, &T ) );

        MBEDTLS_MPI_CHK( mpi_montmul( X, &W[wbits], N, mm, &T ) );

        state--;
        nbits = 0;
        wbits = 0;
    }
}
\end{lstlisting}
\rulebelow
\caption{mbedTLS 2.9.0 - bignum.c lines 1717-1773 (abbreviated)}
\label{mbedtls:squareonzero}
\end{figure}

\begin{figure}
  \begin{lstlisting}
    /*
     * process the remaining bits
     */
    for( i = 0; i < nbits; i++ )
    {
        MBEDTLS_MPI_CHK( mpi_montmul( X, X, N, mm, &T ) );

        wbits <<= 1;

        <@\textcolor{red}{if( ( wbits \& ( one << wsize ) ) != 0 )}@>
            MBEDTLS_MPI_CHK( mpi_montmul( X, &W[1], N, mm, &T ) );
    }
  \end{lstlisting}
  \caption{mbedTLS 2.9.0 - bignum lines 1775-1786}
  \label{mbedtls:wbitshighrisk}
\end{figure}

\subsubsection{BearSSL 0.5}
BearSSL had no positives when setting the exponent and modulus within the
modular exponentiation code to be the tainted values. The library claims to have
constant-time implementations\cite{BearSSLweb}. The mod\_pow function itself had 0 positives,
since there were no secret dependent branches. The library contains very small
files so additional files were linked to get source which closer resembled the
source analyzed in the other libraries. The added source files included the
definitions of the functions called in the mod\_pow function.

A total of 1 positive was found with the additional source code. The branch is
found in the conversion into Montgomery form. The positive is shown on line 6 of
figure \ref{code:bear-tmont}.

\begin{figure}
  \begin{lstlisting}
void
br_i32_to_monty(uint32_t *x, const uint32_t *m)
{
  uint32_t k;

  <@\textcolor{red}{for (k = (m[0] + 31) >> 5; k > 0; k --)}@> {
    br_i32_muladd_small(x, 0, m);
  }
}
  \end{lstlisting}
  \caption{BearSSL 0.5 - br\_i32\_tmont.c}
  \label{code:bear-tmont}
\end{figure}

In this library, no structure represents the big integers, instead an array
serves the purpose. The first element in the array is used to compute the size
and all subsequent elements in the array are the data. The low number of results
here is not due to the field sensitivity, but the ability to set the tainted
variables by function context. The only variables set to be tainted in the
analysis are \codevar{e} and \codevar{m} within the \codefn{br\_i32\_modpow}
function.

The main loop, shown in figure \ref{bearssl:mainloop}, still uses the same square and multiply type method, but
BearSSL uses the square-and-always-multiply method with a constant time
conditional copy to avoid the problem.

\begin{figure}[!htpb]
  \begin{lstlisting}
    for (k = 0; k < ((uint32_t)elen << 3); k ++) {
      uint32_t ctl;

      ctl = (e[elen - 1 - (k >> 3)] >> (k & 7)) & 1;
      br_i32_montymul(t2, x, t1, m, m0i);
      CCOPY(ctl, x, t2, mlen);
      br_i32_montymul(t2, t1, t1, m, m0i);
      memcpy(t1, t2, mlen);
    }
\end{lstlisting}
\caption{BearSSL 0.5 - br\_i32\_modpow.c main loop}
\label{bearssl:mainloop}
\end{figure}


\subsubsection{OpenSSL1.1.0g}

     The number of positives was least when utilizing the field-sensitive
     analysis with a whitelist and the full big number library source. More than
     half the original number of positives were eliminated when using the best
     configuration. There were still remaining false positives, which are a
     result of pointers to different data-types pointing to the same node. In
     this library, the exponent, the plain-text and the modulus and result
     pointers all were represented by the same node. This means that even though
     in practice those pointers point to different memory locations, all were
     treated as the same memory location. The nodes were separate at the
     beginning of the analysis, but the points-to analysis deemed these nodes as
     equivalent, and merged them together in the memory graph.

     There are 5 methods for carrying out the modular exponentiation defined in
     the OpenSSL. There is the reciprocal method, and 3 variations of the
     Montgomery method, and then a simple sliding-window method. The last method
     is unreachable from the \codefn{BN\_mod\_exp} function, so no results are included for
     that method.


\noindent
\textbf{Reciprocal Method}

The reciprocal method has two looped high-risk positives both leaking data
  from the exponent used in the modular exponentiation. The variables are the
  exponent \codevar{p} and the modulus \codevar{m}. Both of these structures have
  the data elements are located as the first element in the structure. The tainted
  variables for this function are \codevar{p} 0 and \codevar{m} 0. The main
  processing loop, shown in figure \ref{code:bnexp-recp}, traverses the bits of
  \codevar{p} and looks for the first set bit, resulting in the first looped
  positive. The function \codefn{BN\_is\_bit\_set} is from \codefile{bn\_lib.c}
  and is linked with \codefile{bn\_exp.c} for the source definition. This means
  that since \codevar{p} 0 is tainted, the \codefn{BN\_is\_bit\_set} function
  result is derived from \codevar{p} 0. From the function definition in figure
  \ref{code:bn-isbitset}, the data array from the bignum structure is indexed
  using some transformation of the input parameter \codevar{n} and then masks all
  but one bit. This function leaks 1 bit of \codevar{p} at a time, but is used
  within a loop, where the input parameter is \codevar{wstart} and changes with
  the loop iteration causing more bits of \codevar{p} to be leaked as the loop
  executes.

  This result is still reported if the \codefile{bn\_lib} source file is not linked with
  the \codefile{bn\_exp.c} file. When the function definition is unreachable, all reachable
  sources from the arguments flow to the return values and other mutable
  values. In this case this means that since \codevar{p} 0 is tainted, the return value
  from the \codefn{BN\_is\_bit\_set} function is also tainted.

\begin{figure}[h!]
\ruleabove
\begin{lstlisting}
for (;;) {
    <@\textcolor{red}{if (BN\_is\_bit\_set(p, wstart) == 0)}@> {
        if (!start)
            if (!BN_mod_mul_reciprocal(r, r, r, &recp, ctx))
                goto err;
        if (wstart == 0)
            break;
        wstart--;
        continue;
    }
   ...
}
\end{lstlisting}
\rulebelow
\caption{OpenSSL 1.1.0g - bn\_exp.c lines 250 - 259}
\label{code:bnexp-recp}
\end{figure}



\begin{figure}[h!]
\ruleabove
\begin{lstlisting}
int BN_is_bit_set(const BIGNUM *a, int n)
{
    int i, j;

    bn_check_top(a);
    if (n < 0)
        return 0;
    i = n / BN_BITS2;
    j = n % BN_BITS2;
    if (a->top <= i)
        return 0;
    return (int)(((a->d[i]) >> j) & ((BN_ULONG)1));
}
\end{lstlisting}
\rulebelow
\caption{OpenSSL 1.1.0g - bn\_lib.c lines 741 - 753}
\label{code:bn-isbitset}
\end{figure}

Within the same processing loop, the second high-risk looped result can be
found, line 6 in figure\ref{code:bn-recp-hi2}. The analysis reports this result
for the same reason as the previous result. The analysis results are only
tracking explicit flow, but an important implicit flow result can be followed
from this positive. Since the loop variable \codevar{i} flows to the window end
variable \codevar{wend} when this branch condition is true, there is implicit
flow between the data in \codevar{p} and the \codevar{wend} variable. The
subsequent for loop then ranges from 0 to \codevar{j} a value computed simply
from \codevar{wend}. The number of iterations for the loop from 0 to \codevar{j}
is then a function of the value of \codevar{p}. In previous versions of OpenSSL,
a cache-based attack identified by observing misses that occurred
\cite{percival2005cache}. The misses allowed the attacker to know whether the
\codefn{mod\_mul} call was executing from line 18 or line 21. Though this code is
reachable, the function should not run if the constant time options are selected
for the exponent and modulus. Though this analysis is done on explicit flow, an
important implicit flow problem exists with flow from line 6 to
\codevar{wvalue}. In the montgomery constant time implementation, a defensive
scatter gather technique is used to mitigate cache-attacks that identify which
cache block was accessed by the index \codefn{wvalue}
\cite{cryptoeprint:2011:239,DBLP:journals/corr/DoychevK16}.


\begin{figure}[h!tpb]
\ruleabove
\begin{lstlisting}
for(;;){
  ...
        for (i = 1; i < window; i++) {
            if (wstart - i < 0)
                break;
            <@\textcolor{red}{if (BN\_is\_bit\_set(p, wstart - i))}@> {
                wvalue <<= (i - wend);
                wvalue |= 1;
                wend = i;
            }
        }

        /* wend is the size of the current window */
        j = wend + 1;
        /* add the 'bytes above' */
        if (!start)
            for (i = 0; i < j; i++) {
                if (!BN_mod_mul_reciprocal(r, r, r, &recp, ctx))
                    goto err;
            }
        if (!BN_mod_mul_reciprocal(r, r, val[wvalue >> 1], &recp, ctx))
            goto err;
   ...
}
\end{lstlisting}
\rulebelow
\caption{OpenSSL 1.1.0g - bn\_exp.c lines 250-297}
\label{code:bn-recp-hi2}
\end{figure}

\noindent
\textbf{Montgomery Method}

   In the high-risk classifications, there is the category of controllable
   branches. These are when there is a branch which the outcome is dependent
   both on user-input as well as secret data. As with the reciprocal method, the
   exponent and modulus, \codevar{p} 0 and \codevar{m} 0 respectively, are tainted. The
   untrusted user specified data is \codevar{a} 0 which is another bignum structure.

   Line 1 in figure \ref{code:mont-nonconst-compare} is controllable, because
   there is a comparison operation done between the plain-text and modulus.
   Assuming \codevar{a} 0 is controllable, then it could be possible to change
   \codevar{a} 0 enough to discover \codevar{m} 0. This branch is done to
   satisfy the assumption that $aa < m$ for the rest of the processing.


\begin{figure}[h!]
\ruleabove
\begin{lstlisting}
<@\textcolor{red}{if (a->neg || BN\_ucmp(a, m) >= 0)}@> {
    if (!BN_nnmod(val[0], a, m, ctx))
        goto err;
    aa = val[0];
} else
    aa = a;

...
if (!BN_to_montgomery(val[0], aa, mont, ctx))
    goto err;               /* 1 */
\end{lstlisting}
\rulebelow
\caption{OpenSSL 1.1.0g - bn\_exp.c lines 363-368}
\label{code:mont-nonconst-compare}
\end{figure}

Line 2 of figure \ref{code:mont-nonconst-compare} calls a function that is not
linked with the analyzed source. One of the results reported from the analysis
is line 374 of \codefile{bn\_exp.c}. This line is reported due to flow from
\codevar{m} to \codevar{val[0]} from the undefined function \codefn{BN\_nnmod}.
Depending on the branch taken the value of \codevar{aa} is equivalent to
\codevar{val[0]} and thus the branch on the \codefn{BN\_to\_montgomery} call is
tainted due to the result of that branch being dependent on tainted data.

The two high-risk looped results are from code identical to the reciprocal
method branching on set bits in p within the main processing loop. These lines
for the Montgomery method are located at 416 and 437 respectively in
\codefile{bn\_exp.c}.

\noindent
\textbf{Constant-Time Montgomery Method}

There is one controllable positive for the constant-time Montgomery method. This
branch is similar to the one found in the Montgomery method, but instead of
calling \codefn{bn\_nnmod} the normal \codefn{bn\_mod} is called. This result is
seen in figure \ref{code:mont-const-compare}. The length of time to execute these
branches is longer for the case where \codevar{a} is greater than 0 and
\codevar{m}.

\begin{figure}[h!]
\ruleabove
\begin{lstlisting}
<@\textcolor{red}{if (a->neg || BN\_ucmp(a, m) >= 0)}@> {
    if (!BN_mod(&am, a, m, ctx))
        goto err;
    if (!BN_to_montgomery(&am, &am, mont, ctx))
        goto err;
} else if (!BN_to_montgomery(&am, a, mont, ctx))
    goto err;
\end{lstlisting}
\rulebelow
\caption{OpenSSL 1.1.0g - bn\_exp.c lines 752-758}
\label{code:mont-const-compare}
\end{figure}

There are still function calls to \codefn{BN\_is\_bit\_set}, but not as branch
conditions. Instead the set bits of the window are extracted, and passed as a
parameter to select the correct value from the pre-computed table of powers.


\begin{figure}[h!]
\ruleabove
\begin{lstlisting}
for (wvalue = 0, i = bits % 5; i >= 0; i--, bits--)
    wvalue = (wvalue << 1) + BN_is_bit_set(p, bits);
bn_gather5(tmp.d, top, powerbuf, wvalue);
\end{lstlisting}
\rulebelow
\caption{OpenSSL 1.1.0g - bn\_exp.c lines 852-854}
\label{code:mont-const-scatter-gather}
\end{figure}

It is possible that there are true positives within the source code given in
figure \ref{code:mont-const-scatter-gather}. This code defines the
\codefn{bn\_gather5} function, but that source was not analyzed. The only
analyzed source files were \codefile{bn\_exp.c} and \codefile{bn\_lib.c}.

\noindent
\textbf{Montgomery Word Method}


Within the Montgomery word method, 1 high-risk looped branch is reported. The
  high-risk looped branch is the same branch from the other methods involving the
  BN\_is\_bit\_set function on the exponent \codevar{p}. This positive is found
  on line 1 of figure \ref{code:mont-word-is-bit-set}.

  \begin{figure}
    \begin{lstlisting}
<@\textcolor{red}{if (BN\_is\_bit\_set(p, b))}@> {
    next_w = w * a;
    if ((next_w / a) != w) { /* overflow */
        if (r_is_one) {
            if (!BN_TO_MONTGOMERY_WORD(r, w, mont))
                goto err;
            r_is_one = 0;
        } else {
            if (!BN_MOD_MUL_WORD(r, w, m))
                goto err;
        }
        next_w = a;
    }
    w = next_w;
}
    \end{lstlisting}
    \caption{OpenSSL 1.1.0g - bn\_exp.c lines 1200-1214}
    \label{code:mont-word-is-bit-set}
  \end{figure}


  \noindent
  \textbf{Simple Method}

  The sliding window attacks from the other libraries were also exploitable
  in earlier version of OpenSSL. The code for the function \codefn{bn\_mod\_exp\_simple} is
  specified as an entry point, although it is not reachable from the
  main entry point of the algorithm. Line 7 in figure \ref{openssl:simple} is reported
  because the branch is dependent on the current bits value of the exponent
  \codevar{p}. This result is directly computed from the exponent and within a
  loop so the amount of leakage is high if the branch outcome is known. The
  timing-channels within this implementation was found in the timing difference
  between the \codefn{mod\_mul} where \codevar{r} is squared
  \cite{percival2005cache}. The implementation of the \codefn{mod\_mul}
  function calls \codefn{bn\_sqr} if both inputs are the same, otherwise the
  more expensive \codefn{bn\_mul} is called.


     \begin{figure}[!htpb]
       \begin{lstlisting}
        j = wstart;
        wvalue = 1;
        wend = 0;
        for (i = 1; i < window; i++) {
            if (wstart - i < 0)
                break;
            <@\textcolor{red}{if (BN\_is\_bit\_set(p, wstart - i))}@> {
                wvalue <<= (i - wend);
                wvalue |= 1;
                wend = i;
            }
        }

        /* wend is the size of the current window */
        j = wend + 1;
        /* add the 'bytes above' */
        if (!start)
            for (i = 0; i < j; i++) {
                if (!BN_mod_mul(r, r, r, m, ctx))
                    goto err;
            }

        /* wvalue will be an odd number < 2^window */
        if (!BN_mod_mul(r, r, val[wvalue >> 1], m, ctx))
            goto err;
       \end{lstlisting}
       \caption{OpenSSL 1.1.0g - bn\_exp.c lines 1326 - 1350}
       \label{openssl:simple}
     \end{figure}

\subsection{Verifying a 2-safety Property}

\begin{table*}[!t]
  \centering
    \ra{1.2}
  \begin{tabular}{@{}lcrrrrcrrrr@{}}
    \toprule
     & & \multicolumn{4}{c}{Ct-verif} & &  \multicolumn{4}{c}{CtChecker} \\
    \cmidrule{3-6} \cmidrule{8-11}
    Library&& Baseline & No Loops & Removed & Diff && Baseline & No Loops & Removed & Diff\\
    \midrule
    Libgcrypt 1.8.2                 &&  49 & 29 & 0 & 20 && 25 & 25 & 1 & 0 \\
    BearSSL 0.5                     &&  14 & 10 & 0 &  4 && 10 & 10 & 0 & 0 \\
    mbedTLS 2.9.0                   && 100 & 80 & 0 & 20 && 79 & 78 & 2 & 0 \\
    OpenSSL 1.1.0g                                                                                 \\
    \hspace{0.25cm}Reciprocal       &&  23 & 19 & 0 &  4 && 19 & 19 & 0 & 0 \\
    \hspace{0.25cm}Mont.            &&  30 & 24 & 0 &  6 && 23 & 23 & 0 & 0 \\
    \hspace{0.25cm}Mont Const. Time &&  39 & 29 & 0 & 10 && 25 & 25 & 3 & 0 \\
    \hspace{0.25cm}Mont. Word       &&  18 & 17 & 0 &  1 && 15 & 15 & 0 & 0 \\
    \bottomrule
  \end{tabular}
\caption{ The baseline file (version 2) accommodated the excluded source.  Positives caused by loops were removed from the baseline in a separate file (version 3). A new file removed the remaining positives (version 4).}
\label{tbl:runtimes}
\end{table*}

In this section, we discuss another approach to verify constant time code,
and compare the precision and accuracy between it and CtChecker. We consider
a recent effort that verifies a 2-property, expressing the security of a
program as a set of logical statements or conditions, and verifying them
automatically using a theorem solver. Our goal was to perform a reasonable
comparison that focuses on the advantages of each method, and not the
engineering details associated with implementing them.

\subsubsection{Background}
The paper of Almeida et al. uses an approach based on a reduction of the
security of a program \textit{P} to the assertion-safety of a program
\textit{Q}, and implements it in a prototype, Ct-verif.
\cite{almeida2016verifying}. The reduction is inspired by prior work
on self-composition and product programs.  A product program \textit{Q}
verifies the security of \textit{P} by simulating two executions in lockstep.
It asserts the equality of each public input with its renamed copy at each
branching instruction. Public outputs are not taken into consideration, and
the technique constructs an output-insensitive product program.  Theoretically,
the approach is sound and complete in that all safe programs have a correct
security verdict, and all unsafe programs have a correct insecurity verdict.

Ct-verif verifies optimized LLVM by implementing their reduction
technique.  The SMACK verification tool is used as a front-end to compile the
annotated C-code via Clang and the resulting compiled and optimized LLVM code
is translated to Boogie code.  The reduction is performed on the Boogie code and
then applied to the Boogie verifier, which uses an SMT logic solver.  Ct-verif
provides an annotation interface for public inputs.

The implementation does not provide results for the entire source. During its
product program verification, if an error state is reached (a violation is
found), no transition is enabled and the system will stop its execution. Constant
time violations that occurred first in sequence were reported, but
identification of violations that occurred afterwards was not guaranteed.
To identify all violations, we replaced each error with some constant time
code and re-ran the tool on the revised code.

We perform a reasonable comparison between the two tools to better understand
the distinct advantages of each method. We aim to avoid the drawbacks of the tools
caused by engineering issues; we assume the best setting for each tool.

\subsubsection{Limitations of Ct-verif}
Although the approach is theoretically sound and complete, the results of
a practical interpretation, must be carefully analyzed.  To perform
a reasonable comparison, we avoid key limitations not present in CtChecker by
revising the analyzed source to accommodate Ct-verif's technical flaws.  For
instance, loops and non-static length arrays are not supported.

Loop invariants are automatically computed to verify program loops.  When
counters are used to index arrays, a proof of security would require Ct-verif
to infer that the memory accesses are in the range of public values.  \sout{Although
there are options to avoid this difficulty, such as enforcing a loop limit or
unrolling a loop, the [actually, why doesn't this work?].}  To avoid this
limitation, loops were replaced with a branch statement that depended on the
values that affected the loop bounds.

Constant sized arrays are supported by adding an extra annotation, but variable
sized arrays are not. To ensure memory accesses are not in the range of secret
values, array indexing needs to be fixed.  We dictated that any array access
was fixed in between these annotated bounds.  False positives caused by loops
or variable array accesses were included in a separate result for Ct-verif. The
revisions made to replace a for-loop and a variable array access in \codefn{MPN\_NORMALIZE}
are shown in figure 30.

We did not include the full source in all of the libraries, except for in BearSSL,
which has a small amount of reachable code from the modular exponentiation function.
Undefined values, such as those that are assigned to the return value of a function
and are excluded from the analysis, were replaced with new variables.  CtChecker handles
excluded source code by conservatively tainting the return value of a function
whose parameter is tainted.  In contrast, Ct-verif taints the return value
regardless of the state of the parameters.  In this direct comparison, we replaced
the excluded function's return value with some new variable.  If the return value
was part of a branch condition, and the function was called with tainted
parameters, this line was counted as a positive for both tools.
In figure 28, \codefn{gcrympih\_lshift} is not linked in the analysis, so its
return value is unknown.  Thus, \codevar{carry\_limb} is undefined before the
branch, and the verifier throws an error.  The revision occurs on line 2 wherein
\codevar{carry\_limb} is given a reserved public input.

\begin{figure}[h!]
\begin{lstlisting}
carry_limb = _gcry_mpih_lshift( res->d, rp, rsize, mod_shift_cnt);
carry_limb = PUBLIC_VALUE;
rp = res->d;
if ( carry_limb )
{
    rp[rsize] = carry_limb;
    rsize++;
}
\end{lstlisting}
\caption{Libgcrypt 1.8.2 - mpi-pow.c lines 717-723}
\label{code:libgcrypt_normalize}
\end{figure}

\subsubsection{Results}
We note that all true positives were reported among both tools, so we
observed comparable accuracy.  Mainly, the results explain a couple of
reasons for the significant difference between the precision observed.
Several versions of the  cryptography algorithms were created before
comparing the precision of the two tools against one another.  There
are three versions (version 2, version 3 and version 4, where version
1 is the original file), each of which was developed from the version
number before it.

First, a baseline version was created to accommodate minor limitations
or differences between the tools, such as the way in which they operate
with some undefined values. Any assignment statement with excluded
source was rewritten to assign the variable to a public or private
input, depending on the function return value. If any of the function’s
parameters were tainted, we conservatively tainted its return value.
Thus, a function call inside of a branch condition was counted as a
positive when any of its parameters were tainted. These positives were
counted the same for each tool, and were only observed in OpenSSL.

The second version was constructed from the first, and it allowed us to
examine the number of positives introduced from computing loop invariants.
Each erroneous loop was replaced, and each variable array access was
changed to a fixed one. Loops were replaced by a model that leaked the
same information as the loop: a \codevar{if (e) then} statement where
\codevar{e} contains the loop bounds. Figure 30 is an example of two
revisions in Libgcrypt, one for a loop and one for an array access.
Ct-verif supports an annotation for a static number of values to be
marked as public.  Variable length arrays are not supported, so variable
array element accesses needed to be removed. Every array index was fixed
within the bounds of the array.

The difference in the number of positives reported by Ct-verif in the
baseline and no loops versions is the number of positives caused by
computing loop invariants only.  The number is a significant portion
of the total positives reported on each version. As expected, CtChecker reported the
same number of positives after removing loops.

The third version was created from the second to remove each erroneous
line reported by Ct-verif, resulting in a subset of the original file
that was free of any constant time violation. This final version allowed
us to evaluate the precision of CtChecker in another way; any positive
reported on this version was exclusive to CtChecker, and was expected
to be a false one.

Ct-verif is flow sensitive; the order of statements in a program may
affect the analysis.  CtChecker's flow insensitivity is one cause of
false positives in the results.

Version 4 was constructed by replacing non-constant time code to
intentionally observe 0 positives by Ct-verif. The positives reported
by CtChecker in this final version was the result two design choices
distinct from Ct-verif. First, CtChecker is a flow insensitive analysis
(the order of statements in a program does not matter). In figure 29,
variable \codevar{i} is not tainted until line 851, but flow
insensitivity causes CtChecker to flag line 1041 since \codevar{i} is
assigned to a tainted value, \codevar{bits}. Second, rather than
tainting a range of values, CtChecker taints the variable pointing to
them. In figure 32, \codevar{ep} is a pointer to sensitive data.

Overall, CtChecker exhibited a large improvement precision over
Ct-verif, a result apparent in the difference between the number of
positives reported by each tool in version 2. CtChecker showed equal
or improved precision over Ct-verif when disregarding significant
limitations, such as loop invariants. This result is apparent in the
difference between the number of positives reported by each tool in
version 3.

\begin{figure}[h!]
\begin{lstlisting}
tmp.d[0] = (0 - m->d[0]) & BN_MASK2;
<@\textcolor{red}{for (i = 1; i < top; i++)}@>
    tmp.d[i] = (~m->d[i]) & BN_MASK2;
tmp.top = top;
...
for (wvalue = 0, i = bits % window; i >= 0; i--, bits--)
\end{lstlisting}
\caption{OpenSSL 1.1.0g - bn\_exp.c lines 741-1041}
\label{code:libgcrypt_normalize}
\end{figure}

\subsubsection{Conclusion}
Perhaps the simple heuristic used for-loop invariant generation is the
primary reason the 2-property analysis saw a lesser precision.  Additionally,
while deductive verification is sufficient for verifying non-trivial programs, like
the ones in this paper, it remains difficult to use. As previously mentioned,
there is no trivial method to interpret verification failures. CtChecker uses
many fewer annotations to run its analysis, and provides a complete list of
violations for each source line.

\begin{figure}[h!]
\begin{lstlisting}
#define MPN_NORMALIZE(d, n)    \
    do {		                   \
      if((n) > 0) /*while( (n) > 0 )*/ { \
        if((d)[0]) /*if( (d)[(n)-1] )*/ \
            dummy++; //break;	 \
        (n)--;	               \
      }		                     \
} while(0)
\end{lstlisting}
\caption{Libgcrypt 1.8.2 - mpi-internal.h lines 113-120}
\label{code:libgcrypt_normalize}
\end{figure}


\begin{figure}[h!]
\begin{lstlisting}
<@\textcolor{red}{if ( rp == ep )}@>
{
    /* RES and EXPO are identical.  Allocate temp. space for EXPO.  */
    ep_nlimbs = esec? esize:0;
    ep = ep_marker = mpi_alloc_limb_space( esize, esec );
    MPN_COPY(ep, rp, esize);
}
\end{lstlisting}
\caption{mbedTLS 2.9.0 - bignum.c lines 1778-1786}
\label{code:libgcrypt_normalize}
\end{figure}


\begin{table*}[!t]
  \centering
    \ra{1.2}
  \begin{tabular}{@{}lcrrrrcrrrr@{}}
    \toprule
    Library && Baseline & No Loops & Removed \\
    \midrule
    Libgcrypt 1.8.2                	    	 && 23 & 18 & 0 \\
    BearSSL 0.5                     	    	 && 16 & 3 & 0 \\
    mbedTLS 2.9.0                          	 && 84 & 67 & 0 \\
    OpenSSL 1.1.0g                                                                                 \\
    \hspace{0.25cm}Reciprocal       	 && 14 & 14 & 0 \\
    \hspace{0.25cm}Mont.            	 && 19 & 16 & 0 \\
    \hspace{0.25cm}Mont Const. Time && 34 & 25 & 0 \\
    \hspace{0.25cm}Mont. Word       	 && 16 & 15 & 0 \\
    \bottomrule
  \end{tabular}
\caption{ The baseline file (version 2) accommodated the excluded source.  Positives caused by loops were removed from the baseline in a separate file (version 3). A new file removed the remaining positives (version 4).}
\label{tbl:runtimes}
\end{table*}

\section{Related Work}
\subsection{Detecting Timing Channels}
   There are several approaches to detecting non-constant time code using both
   static and dynamic taint tracking. In addition to entirely software based
   approaches, hardware language approaches have been used to target leakage from
   low-level hardware features. Efforts have been made to eliminate the
   need for these analyses by creating languages that restrict the amount of
   leakage through covert channels. There are also fuzzing methods which are
   able to detect non-constant time implementations\cite{Somorovsky}.

   There are static analysis tools which will verify that a source is constant
   time given a set of security annotations, such as
   VirtualCert\cite{VirtualCert}. VirtualCert provides a static analysis which
   operates a flow-insensitive type analysis. VirtualCert is used for verifying
   isolation for virtualization systems. Our analysis works in a similar way but
   avoids the need for additional annotations.

   Almeida et al. provided a way to verify programs using deductive verification
   \cite{almeida2016verifying}. These analyses accept or reject programs on the
   basis of being constant-time. In Almeida's work, they have a similar analysis
   which uses the same points-to analysis used in this work, to track memory
   operations. Their work is based on self-composition, which is considered both
   sound and complete, however the analysis is not as they specify sources
   of incompleteness. Their analysis does not provide a field-sensitive
   analysis without additional annotations as ours does.

   Flow Tracker is another static analysis that looks for non-constant time
   implementations due to implicit flow\cite{FlowTracker}. FlowTracker is a
   flow-sensitive analysis that operates on LLVM IR code and requires additional
   annotations. The analysis presented in this paper does not require
   additional annotations to achieve a field-sensitive implicit flow analysis.

   CaSym is a static analysis which looks to find cache-based side channels by
   processing LLVM IR\cite{brotzmancasym}. Brotzman et al. propose different
   cache models to achieve results which are architecture independent and
   explores all execution paths. Their analysis achieves their results by
   assuming a cache model and uses symbolic execution to determine the results.
   Our analysis does not require a cache model and, instead of symbolic
   execution, uses information flow. Since we do not assume a cache model, our
   analysis may miss positives that require details of the cache to
   observe.

   CacheD is a trace-based analysis which identifies cache-based timing-channels
   using taint tracking and symbolic execution\cite{wang2017cached}. The
   strength of CacheD is its ability to identify the location of the
   vulnerability. CacheD does look for differences in the state of the cache,
   which are not accounted for in this work. The analysis in this paper also
   identifies the locations of vulnerabilities for timing channels, but does so
   statically.

   FaCT takes a whole different approach to trying to eliminate non-constant
   time methods\cite{cauligi2017fact}. FaCT is a programming language proposed
   specifically to help developers generate assembly code that is free of timing channels.
   They argue that using C does not help developers safely handle sensitive
   data. This approach requires libraries to be rebuilt in this language. Since
   many libraries make use of C currently, this analysis and analyses previously mentioned are
   still necessary to identify possible timing channels in deployed software.

   Zhang et al. propose software-hardware co-design to address the issue of
   cache-based side channels\cite{zhang2015hardware}. In their work, they show
   that it is possible with a small overhead to enforce control over information
   flow at a hardware level specified in software. Their solution involves
   masking time variations by having a consistent timing regardless of actual
   execution time. Our analysis is attempting to find the code which causes the
   timing variations.

   Fuzzing methods target a somewhat different area than the analysis proposed
   in this paper. Fuzzing methods commonly look to exploit attacks using invalid
   inputs or erroneous execution paths~\cite{sutton2007fuzzing}. Cryptosystem vulnerabilities are often
   not a consequence of exception but one from monitoring execution times and
   control-flow choices. Fuzzing methods may miss results found in other types
   of analyses targeted at identifying timing channels.

\subsection{Known Attack on Cryptosystems}
   The high-risk results identified confirm attacks found in previous papers in
   Libgcrypt and OpenSSL. Kocher showed that various cryptography
   algorithms could be compromised through timing
   attacks\cite{kochertiming}. The modular exponentiation functions were
   implemented using a sliding window method in early versions of the libraries.

   Libgcrypt was used as an example, which showed that due to the ability to
   distinguish the transition from a sequence of squares to a multiply
   execution, a majority of the secret key could be leaked
   \cite{bernstein2017sliding}. This attack exploited the Montgomery reduction
   algorithm and followed each multiplication to mount an amplification
   attack. Though the attack is a cache-based attack, the length of the for-loop
   is not constant time and this result is reported among the high-risk findings
   in this paper.

   OpenSSL has an attack very similar to the Libgcrypt square-and-multiply
   attack, as shown by Percival\cite{percival2005cache}. This showed that
   cache misses could be used to identify the multiplications in the
   square-and-multiply sequences. This vulnerability is a result of non-constant
   amount of squares followed by a multiply, a result reported in our findings.

   OpenSSL had a different attack paper by Brumley and
   Boneh\cite{brumley2005remote}. This paper also showed that timing attacks do
   not need to be mounted from the same machine. This is dangerous because these
   attacks were able to be mounted remotely. Another dangerous find is that even
   with virtual machines used for isolation, the secret key could be extracted
   using the methods they describe. This attack was based on the timing difference
   between comparisons of the current value of the cipher text to one of the prime
   numbers used to compute the modulus. The actual source for this timing attack
   was in a different file than the one analyzed for the results in our paper.
   The same reduction timing attack was found in mbedTLS, shown by Dugaurdin et
   al \cite{mbedtlsreductions,kochertiming}.

   Many of these vulnerabilities lie at the source/IR code level, but the
   attacks need more information to be successful. Hence, it is still valuable
   to know which branches could potentially lead to an exposed attack surface.

\section{Conclusion}
   In this work, the addition of several features to a baseline information-flow
   analysis were tested. Field-sensitivity alone, is usually not enough to
   reduce the number of results. Adding additional source code, since
   field-sensitivity is heavily reliant on the points-to analysis for type
   information and tracking, improved the precision of the analysis. The
   addition of a whitelist to eliminate the results which were deemed safe by
   examination helps to also reduce the list to only positives which are
   not expected.

   The tests run and the effectiveness of field-sensitivity in reducing the
   number of positives, depends on a number of factors. The field-sensitive
   results worked best when adding additional source code. Due to the reliance
   on the points-to analysis to provide information for type and offset
   information, the best results were achieved when enough source was provided
   so that sensitive fields were accessed within the source code.

   This analysis given just a list of sensitive data, produces a list of source
   lines which have conditions based on sensitive data. The analysis produces
   fewer results when compared to the baseline and many of the high-risk results
   are confirmed from previously published attack papers. The libraries that did
   see a reduction in results saw a 40-60\% reduction in results.Using the
   classification methods specified, the number of results can be sorted and
   prioritized to potentially identify non-constant time branches.


   \noindent
   \textbf{Future Work}

   Currently, the classification of the results is done manually, but the
   criteria for the categories are simple. Ideally, an addition to the system
   would automatically classify the results into at least FP, validation, low,
   and high-risk results. This way developers know which results to understand
   first, and then decide whether a positive needs to be addressed or not.

   Additionally, field sensitivity is heavily dependent on the points-to
   analysis used. An understanding of the points-to analysis limitations aids in
   interpreting results. For example, if the points-to analysis used in this
   paper found that type information across function contexts did not line up,
   then the node would have no type-information, meaning all field sensitivity is
   lost for that node.


\bibliographystyle{ACM-Reference-Format}
\bibliography{research}
\end{document}
