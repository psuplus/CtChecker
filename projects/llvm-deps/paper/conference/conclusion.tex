\section{Related Work}
\subsection{Detecting Timing Channels}
   There are several approaches to detecting non-constant time code using both
   static and dynamic taint tracking. In addition to entirely software based
   approaches, hardware language approaches have been used to target leakage from
   low-level hardware features. Efforts have been made to eliminate the
   need for these analyses by creating languages that restrict the amount of
   leakage through covert channels. There are also fuzzing methods which are
   able to detect non-constant time implementations\cite{Somorovsky}.

   There are static analysis tools which will verify that a source is constant
   time given a set of security annotations, such as
   VirtualCert\cite{VirtualCert}. VirtualCert provides a static analysis which
   operates a flow-insensitive type analysis. VirtualCert is used for verifying
   isolation for virtualization systems. Our analysis works in a similar way but
   avoids the need for additional annotations.

   Almeida et al. provided a way to verify programs using deductive verification
   \cite{almeida2016verifying}. These analyses accept or reject programs on the
   basis of being constant-time. In Almeida's work, they have a similar analysis
   which uses the same points-to analysis used in this work, to track memory
   operations. Their work is based on self-composition, which is considered both
   sound and complete, however the analysis is not as they specify sources
   of incompleteness. Their analysis does not provide a field-sensitive
   analysis without additional annotations as ours does.

   Flow Tracker is another static analysis that looks for non-constant time
   implementations due to implicit flow\cite{FlowTracker}. FlowTracker is a
   flow-sensitive analysis that operates on LLVM IR code and requires additional
   annotations. The analysis presented in this paper does not require
   additional annotations to achieve a field-sensitive implicit flow analysis.

   CaSym is a static analysis which looks to find cache-based side channels by
   processing LLVM IR\cite{brotzmancasym}. Brotzman et al. propose different
   cache models to achieve results which are architecture independent and
   explores all execution paths. Their analysis achieves their results by
   assuming a cache model and uses symbolic execution to determine the results.
   Our analysis does not require a cache model and, instead of symbolic
   execution, uses information flow. Since we do not assume a cache model, our
   analysis may miss positives that require details of the cache to
   observe.

   CacheD is a trace-based analysis which identifies cache-based timing-channels
   using taint tracking and symbolic execution\cite{wang2017cached}. The
   strength of CacheD is its ability to identify the location of the
   vulnerability. CacheD does look for differences in the state of the cache,
   which are not accounted for in this work. The analysis in this paper also
   identifies the locations of vulnerabilities for timing channels, but does so
   statically.

   FaCT takes a whole different approach to trying to eliminate non-constant
   time methods\cite{cauligi2017fact}. FaCT is a programming language proposed
   specifically to help developers generate assembly code that is free of timing channels.
   They argue that using C does not help developers safely handle sensitive
   data. This approach requires libraries to be rebuilt in this language. Since
   many libraries make use of C currently, this analysis and analyses previously mentioned are
   still necessary to identify possible timing channels in deployed software.

   Zhang et al. propose software-hardware co-design to address the issue of
   cache-based side channels\cite{zhang2015hardware}. In their work, they show
   that it is possible with a small overhead to enforce control over information
   flow at a hardware level specified in software. Their solution involves
   masking time variations by having a consistent timing regardless of actual
   execution time. Our analysis is attempting to find the code which causes the
   timing variations.

   Fuzzing methods target a somewhat different area than the analysis proposed
   in this paper. Fuzzing methods commonly look to exploit attacks using invalid
   inputs or erroneous execution paths~\cite{sutton2007fuzzing}. Cryptosystem vulnerabilities are often
   not a consequence of exception but one from monitoring execution times and
   control-flow choices. Fuzzing methods may miss results found in other types
   of analyses targeted at identifying timing channels.

\subsection{Known Attack on Cryptosystems}
   The high-risk results identified confirm attacks found in previous papers in
   Libgcrypt and OpenSSL. Kocher showed that various cryptography
   algorithms could be compromised through timing
   attacks\cite{kochertiming}. The modular exponentiation functions were
   implemented using a sliding window method in early versions of the libraries.

   It has been shown, using Libgcrypt as an example, that the ability to
   distinguish a transition from a sequence of squares to a multiply
   execution can leak a majority of the secret key.
   \cite{bernstein2017sliding}. This attack exploited the Montgomery reduction
   algorithm and followed each multiplication to mount an amplification
   attack. Though the attack is a cache-based attack, the length of the for-loop
   is not constant time and this result is reported among the high-risk findings
   in this paper.

   OpenSSL has an attack very similar to the Libgcrypt square-and-multiply
   attack, as shown by Percival\cite{percival2005cache}. This showed that
   cache misses could be used to identify the multiplications in the
   square-and-multiply sequences. This vulnerability is a result of non-constant
   amount of squares followed by a multiply, a result reported in our findings.

   OpenSSL had a different attack paper by Brumley and
   Boneh\cite{brumley2005remote}. This paper also showed that timing attacks do
   not need to be mounted from the same machine. This is dangerous because these
   attacks were able to be mounted remotely. Another dangerous find is that even
   with virtual machines used for isolation, the secret key could be extracted
   using the methods they describe. This attack was based on the timing difference
   between comparisons of the current value of the cipher text to one of the prime
   numbers used to compute the modulus. The actual source for this timing attack
   was in a different file than the one analyzed for the results in our paper.
   The same reduction timing attack was found in mbedTLS, shown by Dugaurdin et
   al \cite{mbedtlsreductions,kochertiming}.

   Many of these vulnerabilities lie at the source/IR code level, but the
   attacks need more information to be successful. Hence, it is still valuable
   to know which branches could potentially lead to an exposed attack surface.

\section{Conclusion}
   In this work, the addition of several features to a baseline information-flow
   analysis was tested. On its own, field-sensitivity is usually not adequate to
   reduce the number of results. Adding additional source code improved the precision of the analysis since
   field-sensitivity is heavily reliant on the points-to analysis for type
   information and tracking. The
   addition of a whitelist to eliminate the results deemed safe by
   examination also helps reduce the list to only positives which are
   not expected.

   Tests show that and the effectiveness of field-sensitivity in reducing the
   number of positives depends on a number of factors. The field-sensitive
   results worked best when adding additional source code. Due to the reliance
   on the points-to analysis to provide information for type and offset
   information, the best results were achieved when enough source was provided
   so that sensitive fields were accessed within the source code.

   This analysis, given only a list of sensitive data, produces a list of source
   lines which have conditions based on sensitive data. The analysis produces
   fewer results when compared to the baseline, and many of the high-risk results
   are confirmed from previously published attack papers. The libraries that did
   see a reduction in results saw a 40-60\% reduction in results. Using the
   classification methods specified, the number of results can be sorted and
   prioritized to potentially identify non-constant time branches.


   \noindent
   \textbf{Future Work}

   Currently, the classification of the results is done manually, but the
   criteria for the categories are simple. Ideally, an addition to the system
   would automatically classify the results into at least FP, validation, low,
   and high-risk results. This way developers know which results to understand
   first, and then decide whether a positive needs to be addressed or not.

   Additionally, field sensitivity is heavily dependent on the points-to
   analysis used. An understanding of the points-to analysis limitations aids in
   interpreting results. For example, if the points-to analysis used in this
   paper found that type information across function contexts did not line up,
   then the node would have no type-information, meaning all field sensitivity is
   lost for that node.
