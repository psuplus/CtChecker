\section{\sysname{} Design}

In this section, we describe the design details of \sysname{}.

In this work, adjustments were made to improve the precision of a baseline
taint analysis. The baseline analysis was proposed by Moore et al
\cite{moore2011static}. Adding field-sensitivity and adding a whitelist both
improved precision. Adding field-sensitivity aided in more
accurately representing flow within structures and arrays. The baseline could
not track flow to members of a structure, only flow to the structure itself. The
field-sensitive analysis is the major change to the baseline system. This
section will outline the three sections which make up the analysis. The
information flow capture parses the IR code to identify the values from which
information flows. The constraint generation is done by interpreting the
captured flows and creating constraints. Lastly, to identify secret-dependent
branches, the least solution is found and branch conditions are checked to see if
sensitive information was used to compute the condition.

\subsection{Generating Information Flows from IR Code}

   Given a program in the LLVM IR representation, flow of information is tracked
   by identifying the operands. Each operand can be considered as a source or a
   sink, though this alone is not enough information when considering the
   effects of pointers. Since pointers can point to different data and the data
   to which they point can change, the points-to analysis is used. The
   points-to analysis is used to keep track of which memory locations are
   associated with the LLVM IR values identified as sources or sinks.

   For each LLVM value, there are 3 elements which can be constrained. There are
   3 transformation functions which return the element to be constrained for
   that value. For any operation, zero or more of the three functions (V, D, R)
   may be used in generating a constraint for the instruction. Forming a
   constraint requires a source constraint element and a sink constraint
   element. The flows identified from the instructions create a constraint by
   adding a rule, which states that the source constraint element flows to the sink
   constraint element.

   \begin{itemize}
   \item $V(x)$ is the constraint element associated with the IR value $x$.
   \item $D(x, offset, size)$ is the constraint element with the appropriate
     type $size$ associated with the memory represented by IR value $x$ and the
     $offset$.
   \item $R(x, size)$ is the set of all reachable memory locations that are
     represented by IR value $x$.
   \end{itemize}

   The baseline analysis used these same transformations, but the addition
   of field sensitivity changed the behavior of the functions $D$ and $R$, and the
   memory nodes were only able to be constrained as a single entity.
   Since a whole data structure was referred to as a single element, tracking
   the flow to individual fields was not available. The \textit{DSA} analysis
   provides a type map, which can be used to identify the field type and offset
   within the data structure. Using this information, the $D$ and $R$ functions
   consumed an extra operand to return the element to be constrained for a
   particular field.

   The $R$ transformation is similar to the $D$ transformation in that it
   returns the constraint element relevant to the memory addressed, but the $R$
   transformation is used in the case of external functions or source. The $R$
   transformation is the more conservative transformation, returning all possible
   constraint elements tied to a pointer, so the offset and size are not
   required like they are for the $D$ transformation. In the baseline, the $D$
   transformation did not exist, thus this transformation was added to achieve
   field-sensitivity.

   For most IR instructions, explicit flow is constrained by adding the
   constraint $V(source) \rightarrow V(sink)$. There are a subset of LLVM IR
   memory operations that require interaction with \textit{DSA}. The
   store, load and GetElementPtr (GEP) instructions are the most important for
   addressing the issue of field sensitivity. The rules for how these are
   constrained are given in figure \ref{fig:llvm-mem-ir}.

\begin{figure}[h!]
  \hrulefill
  \begin{lstlisting}[style=nolinenumberstyle]
<result> = alloca <type> [, <ty> <NumElements>] [, align <alignment>]
  \end{lstlisting}
  \explicitflow{ $\rightarrow V(\result)$}

  \implicitflow{$V(PC) \rightarrow V(\result)$}

  \begin{lstlisting}[style=nolinenumberstyle]
<pointer> = getelementptr inbounds <ty>, <ty>* <ptrval>{, [inrange] <ty> <idx>}*
  \end{lstlisting}
  \explicitflow{ $V(\ptrval) \rightarrow V(\pointer)$}

  \implicitflow{ $V(PC) \cup V(\ptrval) \rightarrow V(\pointer)$}

  \begin{lstlisting}[style=nolinenumberstyle]
store <ty> <value>, <ty>* <pointer>[, align <alignment>]
  \end{lstlisting}
  \explicitflow{$V(\myvalue) \rightarrow D(\pointer, \offset)$}

  \implicitflow{$V(PC) \cup V(\pointer) \rightarrow D(\pointer, \offset)$}

  \begin{lstlisting}[style=nolinenumberstyle]
<result> = load <ty>, <ty>* <pointer>[, align <alignment>]
  \end{lstlisting}
  \explicitflow{ $V(\pointer) \cup D(\pointer,\offset) \rightarrow V(\result)$}

  \implicitflow{ $V(PC) \cup V(\pointer) \cup D(\pointer,\offset) \rightarrow V(\result)$}
  \hrulefill
  \caption{LLVM Memory IR Flow Rules}
  \label{fig:llvm-mem-ir}
\end{figure}

The load and the store instructions are the instructions which changed the most
from the baseline. In the baseline, these rules were the same, but the offset
was not considered in the $D$ transformation. The store and load instructions do
not have an offset as a operand in their instructions. The offset is found  by
examining the IR instruction which yielded the operand. If the operand is a pointer
computed from a structure or an array, then the pointer will have
been calculated using the GEP instruction. By including the GEP instruction in
the sources for the load (and the sink for the store), the instruction can be
analyzed and the index can be accessed.


\subsubsection{Constraint Generation}

For operations which do not modify memory, the constraints are generated such
that $V(inputs) \rightarrow V(outputs)$. Loads and store instructions require
additional steps to include the points-to analysis. The points-to analysis
represents memory as nodes in a graph. Structures and arrays are represented as
a single node with as many types and offsets as necessary. Figure
\ref{fig:storeconstraint} shows how a store would be constrained using
additional information from the pointer operand and the type information from
the points-to analysis.


\begin{figure}[h!]
\begin{lstlisting}
typedef struct {
  int rd_key;
  int rounds;
} AES_KEY;

AES_KEY pk;
pk.rounds = 3;
\end{lstlisting}

\begin{lstlisting}
  %pk = alloca %struct.AES_KEY, align 4
  %rounds = getelementptr inbounds %struct.AES_KEY, %struct.AES_KEY* %pk, i32 0, i32 1
  store i32 3, i32* %rounds, align 4
\end{lstlisting}
  \caption{Store Constraint Example}
  \label{fig:storeconstraint}
\end{figure}

The $D$ transformation utilizes the points-to analysis to retrieve the correct
memory node. The memory node stores type information for any types accessed from
that node in the IR code. For stack arrays and structures, the type information
from the points-to analysis is used to create individual elements for each of
the members accessible from that structure/array. Thus the offset must be passed
to $D$ along with the pointer to select the correct element. For structures, the
instructions only give an index, so the byte offset is computed using the data
layout information that the points-to analysis uses to create its type information.

In figure \ref{fig:storeconstraint} The \codevar{AES\_KEY} structure has two
fields. The GEP instruction is used to calculate the address of the fields when
operating on the structure. The inputs are the base pointer and zero or more
offsets. The store instruction is constrained using the pointer and the offset
operands, which are provided by the GEP instruction that flows into the store
instruction. In this example, the offset to be used is the last operand. The
last operand is not a byte offset from the base pointer, its an index. Using the
points-to analysis, the index is converted into a byte offset. Bytes are used
instead of the index to account for a types width. The type width matters in the
cases where types are overlapping.

 The rest of this section shows how the analysis handles when the necessary
 information is not available to create a proper field sensitive constraint.

\subsubsection{Factors Affecting Field-sensitivity}
\begin{itemize}
\item{Collapsed Memory Nodes}

   The points-to analysis is type-safe, marking nodes as collapsed when the type
   information is inconsistent. If a node is collapsed, only one constraint
   element is created for that node. Multiple constraint elements for a single
   memory node are only created when the points-to analysis has type information
   available for the node.

\item{Incomplete Type Information}

   The points-to analysis used only provides type information for the values used
   in the analyzed source. The field-sensitive analysis should also be
   conservative when analyzing unknown source. For this, the analysis relies on
   the data layout of structures which is the same information the points-to
   analysis uses to build its type information. The layout information helps in
   the case where a tainted field is not used within the provided source but its
   data structure is passed as an argument to an unknown function. This provided
   the possibility for the tainted field to leak information.

\item{Calls to Unknown Functions}

For functions which the source code is not provided, the constraint rules define
how to treat values and their flow. The rules used, allow for flow between all
pointer types, and also flow of information within the pointer itself. That is,
a structure may only have a single tainted field, but after passing through an
unknown source, all fields are tainted due to possible flow between fields.
Likewise, flow from one pointer to another ensures that any tainted value
propagates in the constraints to other parameters.

\[V_{args} = \bigcup_{i=0}^N V(arg_i)\]
\[R_{args} = \bigcup_{i=0}^{N}\{R(arg_i) : type(arg_i) = pointer\}\]

\noindent
\textbf{Explicit:} $V_{args} \cup R_{args} \rightarrow V(retval) \cup R_{args}$

\noindent
\textbf{Implicit:}$V(PC) \cup V(functionptr) \cup V_{args} \cup R_{args} \rightarrow V(retval) \cup R_{args}$
\end{itemize}

\subsection{Solving Information Flow Constraints}
Given the set of constraints for the program being analyzed, the least solution
where all the sinks are greater than or equal their sources is found, otherwise
known as a least solution. Given a set of initial sensitive values, any sink
which has a value greater than or equal to that value, is considered sensitive.
In the case of cryptosystems, it can be helpful to find all the values which
may have been computed from confidential data. These values are possible
locations for leakage of confidential information, presenting a possible
vulnerability.


Consider figure \ref{alg:simpleflow} for example of how the analysis functions.
First the flow of information must be identified. For simplicity, this example
will just consider the source code, not the instructions. In figure
\ref{alg:simpleflow}, $k$ is just a constant set to some confidential value, no
information flows there. Next, $d.x$ is calculated by adding 1 to the value of
$k$, so there is flow from $k$ to $d.x$. With $d.y$ and $a$, in lines 3 and 4
respectively, they are set with constants, so no flow happens there. Then, a
branch is encountered where the value of $a$ is modified based on the values of
$d.y$ and $k$. Let the branch condition be $b$, there will be flow from $d.y$
and $k$ to $b$. Lastly due to the branch, there is an implicit flow from $d.y$
to $a$ and $k$ to $a$.

For explicit flow the set of constraints is as follows.
\[
  k \leq d.x
\]
\[
  d.y \leq b
\]
\[
  k \leq b
\]
In this case it is easy to see that $d.x$ should be at least as confidential as $k$
since it is directly computed from $k$. Similarly, the branch condition $b$ is
confidential since it must be at least as confidential as $k$. The value $d.y$
is not confidential, but is irrelevant because confidential information is used
to determine the outcome of the branch.

If implicit constraints are to be considered then the set is as follows.
\[
  k \leq d.x
\]
\[
  d.y \leq b
\]
\[
  k \leq b
\]
\[
  b \leq a
\]

In this case the all the constraints are the same apart from the addition of the
last constraint. There is another constraint between the branch condition and a,
because after the branch, the value of a is dependent on the outcome of the
branch. If an attacker had a method of inspecting the value of $a$, they would
learn information about $b$ which is derived from $k$.

The baseline analysis generates a set of constraints similar to the examples
above, but with some imprecision. During the stage of generating constraints
from the information flow, the points-to analysis is leveraged to figure out
what is being pointed to by the variable $d$ to subsequently find the correct
instance of $x$. The points-to analysis will represent the data structure
pointed to by $d$ with one node in the graph, and the constraint generated
refers only to that node, instead of to the specific field in the node. The
effect on the constraints is as follows for explicit flow.

\[
  k \leq d
\]
\[
  d \leq b
\]
\[
  k \leq b
\]

The result is that $d.x$ and $d.y$ are indistinguishable from each other and the
results will report both $d.x$ and $d.y$ as confidential even though manual
scanning of the source shows that $d.y$ is never calculated from confidential
data. This is a common problem when analyzing sources. Take this implementation
of AES, where the data structure used is comprised of both confidential and
public data. The \codevar{AES\_KEY} struct has the confidential \codevar{rd\_key} and a second field
rounds that is considered to be public. Figure \ref{alg:aesstruct} shows
where the inaccuracy can be found in actual source code.

So with the improved analysis, the constraints are generated such that compound
data structures with multiple nodes are able to be constrained separately. Once
the set of constraints for a programs source code is created, the least solution
is found and the results show the line number where the branch occurs. This list
of results shows the branches which are tainted and untrusted.

\begin{figure}[h!]
\begin{lstlisting}
struct AES_KEY_t {
  unsigned * rd_key;
  int rounds;
};

typedef AES_KEY_t AES_KEY;

void AES_enc(char * in, char * out, AES_KEY) {
  unsigned * rk = key->rd_key;
  if (key->rounds > 10) {
    ...;
  }
}
\end{lstlisting}
\caption{Public and private data in structure}
\label{alg:aesstruct}
\end{figure}

\subsection{Result Classification}

The analysis provides a list of source code lines which are vulnerable. Based on
the contents of the input files, a branch is flagged as vulnerable if the
condition depends on data which has been marked as tainted or untrusted. The
reported lines can then be reviewed and sorted to rank them in order of
severity. Classification of the vulnerable branches consists of three stages,
the first removes error-handling results, the second sorts the remaining results
into either a high-risk or low-leakage sets, and the final sorts the high-risk results
depending on the surrounding source context.

Stage 1 is a filter which removes the error handling and input validation
results. For normal operation of a program, these branches check sensitive data
but allow an early exit if the branch condition evaluates to true. In figure
\ref{alg:stage0src}, the variable \texttt{N} is tainted, so the branch is
vulnerable due to the condition depending on \texttt{N} and the data held within
\texttt{N}. The result of this branch is not helpful in the case of an attacker
attempting to learn the value of \texttt{N}. As long as the input is valid, the
condition of this branch will always evaluate to false. If the branch leads to
exit code, it is a candidate for removal in this stage. To conclude stage 1, if
there are results which are reported that leak non-sensitive data, they may be
whitelisted. The whitelist allows results to be ignored, and not propagate taint
as a result in the later stages.

\begin{figure}[h!]
\begin{lstlisting}
if( mbedtls_mpi_cmp_int( N, 0 ) <= 0 || ( N->p[0] & 1 ) == 0 )
    return( MBEDTLS_ERR_MPI_BAD_INPUT_DATA );
\end{lstlisting}
\caption{Error-handling Source Code}
\label{alg:stage0src}
\end{figure}

Stage 2 operates on the remaining results to sort them into high and low-risk
categories. A high-risk branch will be due to one of the operands being either
directly related to the sensitive data or derived from more than 1 bit of
sensitive data. Low-risk branches, are branches which the operand is based on 1
bit of the key or the same operand could derived from some set of sensitive data
values.

\begin{figure}
\begin{lstlisting}
#define MPN_NORMALIZE(d, n) \
do {                        \
  while( (n) > 0 ) {        \
    if( (d)[(n)-1] )        \
      break;                \
    (n)--;                  \
  }                         \
} while(0)

MPN_NORMALIZE(ep, esize);
if (esize * BITS_PER_MPI_LIMB > 512)
  W = 5;
\end{lstlisting}
\caption{High-Risk and Low-Risk Branches}
\label{alg:branchriskexample}
\end{figure}


Figure \ref{alg:branchriskexample} is a snippet of some of the modular
exponentiation code from Libgcrypt 1.8.2. For instance, say the values pointed
to by \texttt{ep} are marked as tainted at the start of the analysis, and the
analysis reports that lines 10 and 11 are vulnerable. Line 10 is the macro
defined in the lines 1-8. Line 10 is reported due to the macro having the
\texttt{if} branch directly dependent on the data pointed to by \texttt{ep}.
Since the branch is directly dependent on the tainted data that branch is
considered high-risk. Line 11 however is a branch based on \texttt{esize}, the
value begin modified within the \texttt{MPN\_NORMALIZE} macro. The value of
\texttt{esize} is not unique to the tainted data, meaning there exists a set of
tainted values which may yield the same \texttt{esize}. If a value is derived
from tainted data, as \texttt{esize} is, but the result is not unique to that
instance of tainted data, it is considered low-risk.

If for any reason, the classification is unclear, such as a mutable type being
passed to source code that is not analyzed, the data is treated as high risk.
This is done in order to be conservative. Further categorization can be
performed on the high-risk set of results in Stage 3. Additionally, if there are
a large number of low-risk result, Stage 2 can be repeated, by appending the
low-risk variables to the whitelist.

Stage 3 specifically looks a the context of the high-risk results from Stage 2.
Each of the results will be classified as one of the following: a single branch, repeated
branch, controllable branch or both (e.g. both controllable and repeated).
Single branches are branches which are not within a loop. Repeated branches are
branches that are part of a loop definition or within the body of a loop.
Controllable branches are that which the branch result is both untrusted and
tainted. The purpose of sorting results this way is to be able to prioritize
which vulnerable branches to examine first when attempting to make the
application secure.
