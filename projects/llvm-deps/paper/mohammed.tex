\documentclass[11pt,a4paper]{article}

%\usepackage[margin=0.75in]{geometry}
\usepackage{algpseudocode}
\usepackage{algorithm}
\usepackage{listings}
\lstset{numbers=left, numberstyle=\tiny, breaklines}
\title{CSE 597 Homework 1\vspace{-1ex}}
\author{Adam Mohammed}
\date{\vspace{-1ex}January 29, 2018}

\lstset{language=c}
\begin{document}
\section{Introduction}
%
\section{Background}
\subsection{Information Flow Analysis}
Information flow analysis tracks interactions of information throughout a
program. A simple application of this analysis is used in LOMAC to restrict
access to data of various integrity levels. LOMAC maintains a concept of
information integrity for each object, and restricts information from flowing
from low integrity objects to high integrity objects. If a program source is
inspected, and data is given a level of integrity or confidentiality then by
following the flow of information in the program, it is possible to track which
data have been computed from confidential or high integrity information. In
cryptographic systems, the encryption keys are confidential, if other data
within the program is calculated from the encryption key, there is a flow of
information from the confidential data to the computed data. Compilers also use
information flow for optimization purposes. By identifying what information is
used and when, alterations to a program can be made to improve cache locality,
hide memory latency and improve performance in other manners. For program level
security, a compiler could be able to check the level of integrity for any
variable through the use of information flow. By tracking confidential or
sensitive pieces of information, any variable or decision which is computed from
a confidential data has a flow from the confidential data to the variable or
branch. An attacker may be able to exploit parts of a program which are
dependent on some confidential data, and with sufficient flow, can possibly
reconstruct the confidential data. An example is given in Algorithm
\ref{alg:simpleflow} to illustrate this concern.

\begin{algorithm}
  \caption{Simple Information Flow}
  \label{alg:simpleflow}
\begin{algorithmic}
  \State $k =$ sensitive information
  \State $d.x =  k + 1$
  \State $d.y = 0$
  \State $a = -1$

  \If {$d.y == k/2$}
    \State $a = 4$ 
  \EndIf

  % \If {$d.x == 2$}
    % \State $b = 2$ 
  % \Endif
\end{algorithmic}
\end{algorithm}

In this simple example $k$ is confidential data, through information flow we
can see that the value of $d.x$ is directly computed from $k$. The value $d.y$
however is not confidential due to its value having no reliance on the
confidential data. The flow between $k$ and $d.x$ is called explicit flow,
since the value of $d.x$ is computed from the confidential data. As the
analysis continues, any value which is computed from $d.x$ will also be
treated as confidential data. There is also implicit flow which can be seen in
the first branch on $d.y$ where it is compared with some value computed from
$k$. In this case $d.y$ is still considered non-confidential data, since the
value has no relation to $k$, but there is a flow from $k$ to $a$ indirectly.
Although, $a$ is not computed from $k$, after the branch executes, the value
of $a$ may or may not have changed. If it does change, the value of $k$ has been
determined. If it does not change, the value of $k$ is not determined, but a
possible value has been ruled out. This indirect flow from $k$ to $a$ is what
is called implicit flow. Tracking the flow of information between variables can
be done by creating a set of constraints for each instruction to represent the
direction of flow. Creating these constraints with mutable pointers, requires
more information than is provided from IR code from the compilation process. 

\subsection{Points-to Analysis}

A Points-to analysis is a reference tracking analysis to identify the targets of
pointers in a program. Information flow alone is not enough once pointers are
involved. For example, in Algorithm \ref{alg:simpleflow}, it must be possible to
identify from the source the location of $d$, and subsequently the location of
$x$. The points-to analysis provides this information so that the proper
location of $x$ and $y$ can be known even if the value of $d$ changes, meaning
it points to another location in memory. In addition to providing spatial
information, the points-to analysis used in this research provides type and data
layout information. The analysis accomplishes tracking by maintaining a internal
graph representation of memory throughout the program. The graph is made up of
nodes which are units of memory that are able to be the target of a pointer.
Data structures identified in program source have one node for an instance of
the data structure, where all fields pertaining to that data structure are
represented by that node. The edges in the graph are references, so it is
possible to identify where the target of a pointer resides. Additionally, this
analysis provides type information along with offsets of the type from the base
pointer. In the baseline tainted flow analysis, the granularity of the results
was achieved by creating  constraints of information flow between nodes from the
points-to graph. This however, as will be discussed, leads to an
over-approximation of the resulting tainted values. The points-to analysis used
also has a method of determining the validity of the type information. 

\subsection{Timing Channels in Crypto-systems}

A timing channel is a side channel in which an attacker uses execution time to
learn information about sensitive data. In some implementations of sliding
window exponentiation, the sequences of squares and multiplies could be measured
due to differences in each methods execution time. This attack though timing
based can be observed in source code in OpenSSL's previous implementation of
sliding window exponentiation, shown in Algorithm \ref{alg:timing_sqr_mlt}. 

\begin{algorithm}
  \caption{Square and Multiply Timing Channel}
\begin{lstlisting}
for (i = 1; i < bits; i++) {
    if (!BN_sqr(v, v, ctx))
        goto err;
    if (BN_is_bit_set(p, i)) {
        if (!BN_mul(rr, rr, v, ctx))
            goto err;
    }
}
\end{lstlisting}
\label{alg:timing_sqr_mlt}
\end{algorithm}

The for-loop here iterates through the number of bits in the confidential power
\texttt{p}, each time through squaring \texttt{v} until a set bit is encountered
in \texttt{p}. Once a set bit is encountered, an additional multiply step is
executed before proceeding to the next loop iteration. This lends itself to a
timing attack due to multiplication more expensive to compute than a square. The
timing channel exists because the number of operations executed during one
iteration changes based on the value of the confidential data. The result is if
one can determine the locations of the multiplies and the length of time
expected between each multiplication, the key can be rebuilt as has been done
by Bernstein et al \cite{bernstein2017sliding}. There are other forms of timing
channels which may or may not be viewable in source code, such as disk access
timing or network timing attacks. 

% More on timing channels?

\section{Improving Field Sensitivity in Taint Analysis}
Improvements to the precision of the baseline taint analysis. Improvements in
precision also increased the accuracy of the analysis by reducing the number of
false positives from the final results. In order to understand the improvements,
the baseline taint analysis and its implementation must be understood. The
analysis breaks down simply in to three component parts, identifying information
flows, generating a constraint set to represent the information flow, and
finally solving the set of constraints.

The first component of the analysis which identifies the information flows of
the program, works on the instruction level. For each program the analysis will
create two sets of constraints, one for explicit information flow, and another
which includes implicit flow. The program attempts to start from the main
function, but if it does not exist all functions will be analyzed the start
point. Any function which calls other functions within its definition, is
analyzed in its own context, so information can flow through the arguments as
well. Each contexts instructions are parsed to identify operands and operation.
The operands are divided into source and sink operands and based on the
operation. The flow is added to the corresponding set based on whether or not
flow is explicit or implicit.

Once the flows through the program are identified, each constraint which makes
up the set, is generated by iterating through the flows. A constraint is added
to the set constraining the sink element in relation to the source element. In a
taint/confidential analysis, the sink must be at least as confidential as the
source in the final result. The outcome of this is if information is
confidential and flows to another value, that information is also considered
confidential. After all the flows are constrained, the set is complete and a
solution is found.

Given the set of constraints for the program, the least solution is found. From
the solution each sink operand will have a value is greater than or equal to
that of the initial confidential data, then that value is identified as being
confidential as well. In the case of crypto-systems, it can be helpful to find
all the values which may have been computed from confidential data. These values
are possible locations for leakage of confidential information, and an attacker
may be able to exploit the computed confidential data to learn information or
reconstruct the original confidential data.

Consider Algorithm \ref{alg:simpleflow} for example of how the analysis
functions. First the flow of information must be identified. For simplicity,
this example will just consider the source code, not the instructions. The more
detailed analysis will be done in the implementation section. In Algorithm
\ref{alg:simpleflow}, $k$ is just a constant set to some confidential value, no
information flows there. Next, $d.x$ is calculated by adding 1 to the value of $k$,
so there is flow from $k$ to $d.x$. With $d.y$ and $a$, in lines 3 and 4 respectively,
they are set with constants, so no flow happens there. Then, a branch is
encountered where the value of $a$ is modified based on the values of $d.y$ and
$k$. Let the branch condition be $b$, there will be flow from $d.y$ and $k$ to
$b$. Lastly due to the branch, there is an implicit flow from $d.y$ to $a$ and $k$ to
$a$.

For explicit flow the set of constraints is as follows.
\[
  k \leq d.x
\]
\[
  d.y \leq b
\]
\[
  k \leq b
\]
In this case it is easy to see that $d.x$ should be at least as confidential as $k$
since it is directly computed from $k$. Similarly, the branch condition $b$ is
confidential since it must be at least as confidential as $k$. The value $d.y$
is not confidential, but is irrelevant because confidential information is used
to determine the outcome of the branch. 

If implicit constraints are to be considered then the set is as follows.
\[
  k \leq d.x
\]
\[
  d.y \leq b
\]
\[
  k \leq b
\]
\[
  b \leq a
\]

In this case the all the constraints are the same apart from the addition of the
last constraint. There is another constraint between the branch condition and a,
because after the branch, the value of a is dependent on the outcome of the
branch. If an attacker had a method of inspecting the value of $a$, they would
learn information about $b$ which is derived from $k$.

The baseline analysis generates a set of constraints similar to the examples
above, but with some imprecision. During the stage of generating constraints
from the information flow, the points-to analysis is leveraged to figure out
what is being pointed to by the variable $d$ to subsequently find the correct
instance of $x$. The points-to analysis will represent the data structure
pointed to by $d$ with one node in the graph, and the constraint generated
refers only to that node, instead of to the specific field in the node. The
effect on the constraints is as follows for explicit flow.

\[
  k \leq d
\]
\[
  d \leq b
\]
\[
  k \leq b
\]

The result is that $d.x$ and $d.y$ are indistinguishable from each other and the
results will report both $d.x$ and $d.y$ as confidential even though manual
scanning of the source shows that $d.y$ is never calculated from confidential
data. This is a common problem when analyzing sources. Take this implementation
of AES, where the data structure used is comprised of both confidential and
public data. The AES\_KEY struct has the confidential rd\_key and a second field
rounds that is considered to be public. Algorithm \ref{alg:aes_struclt} shows
where the inaccuracy can be found in actual source code.

\begin{algorithm}
  \caption{Public and private data in structure}
  \label{alg:aes_struclt}
  \begin{lstlisting}
  struct AES_KEY_t {
    unsigned * rd_key;
    int rounds;
  };

  typedef AES_KEY_t AES_KEY;

  void AES_enc(char * in, char * out, AES_KEY) {
    unsigned * rk = key->rd_key;
    if (key->rounds > 10) {
      ...;
    }
  }
  \end{lstlisting}
  %\lstinputlisting{examples/aes.c}
\end{algorithm}

  In this work, improvements were made over a baseline taint analysis. The
  original taint  
  The changes made since the previous versions of the pass were done to improve
  the sensitivity. The system works by generating constraints for each value that
  is extracted from the control flow of a program. By creating the constraint
  pairs, tracking tainted values and their effect on  the rest of the program
  becomes a problem of tracking the least upper bound for each value.  Originally,
  there was a mapping between values and AbstractLocs which are the data structure
  used to represent the organization of memory. For user-defined types, such as
  classes and structures, one AbstractLoc represented the memory created from
  defining an object from one of the types. consequently, as the original design
  worked just one constraint element mapped to the AbstractLoc despite one
  AbstractLoc representing more than one data field. As a result, if any part of
  the AbstractLoc had a least upper bound that was the same as the tainted
  variable all related fields were considered tainted.

  For this work, changes were made to a compile-time analysis to improve
  the precision. The analysis is built to help the developer find and identify
  places where secret information may leak to an adversary. By tracking the
  information flow during the optimization stage of compiling, the analysis can
  reduce the work the developer needs to do to ensure information integrity is
  maintained. The analysis to provide this level of assurance relies on being
  able to model the memory and accesses to it in conjunction with the source
  and control-flow graph representations of the program. The granularity
  that the analysis can provide can change the number of reported results
  considerably. Before the changes the granularity of the analysis was done
  by object (or structure), meaning if an object at any point had information
  flow from a sensitive piece of information, all data in that object would
  now be considered sensitive. The major change addressed in this work is to
  improve the granularity of the analysis. The motivation for this change is
  to reduce the number of positives that result only as a consequence of the
  poor granularity.


  Algorithm \ref{alg:simpleflow}, let the variable $k$ be a sensitive variable in the following program. Also, let
  variable $d$ be a structure which contains two fields $x$ and $y$. 

  \begin{algorithm}

  \end{algorithm}

  At the start, the $k$ variable holds sensitive information and any flow that has k
  in its computation should be sensitive. In line 2, $k$ influences the value of $d.x$
  and as expected $d.x$ is sensitive. Ideally $d.x$ is the only variable other
  than $k$ that is reported from the analysis. In the original implementation,
  the compiler uses a memory representation that represents the memory that would
  be associated with $d$ and because of the object-level granularity, instead of
  $d.x$ being the only variable that is sensitive from $d$, $d.y$ is also listed
  as a sensitive variable. This simple case results yields a very conservative
  bound on what variables of the program have actually been reached by a sensitive
  piece of information. The draw back from being this conservative is that many of
  the positives that are reported are just a result of being in the same memory
  structure element as the actual variable which has been computed from a
  sensitive piece of information.

  In terms of correctness, $d.y$ has actually never been computed from any
  sensitive information and thus should not be a tainted variable. False
  positives like this propagate throughout the program and for larger programs
  account for a significant amount of the results. The fix to this issue
  requires there to be work done with how constraints are created from elements
  which share the same data type and the same memory element that represented
  that variable in the analysis. First, the issue is to understand how the
  analysis as it is done in the original work functions, and then the necessary
  changes can be addressed.

  The analysis runs as one of the middle layers of compilation. The program
  traverses through and decodes the operands from each instruction that
  generates the flow of information between variables in the program.

  Establishing constraints is done by traversing through the IR and creating a
  set of inequalities between the the source and sink values for each
  instruction. Mathematically, if information flows from a source to a sink, the
  sink must be considered at least as sensitive as the components of which it is
  computed. Thus any information that is considered sensitive is given a high
  starting sensitivity value to force any information it touches to also raise
  the sensitivity value of any variable which is computed from it. Once the set
  of constraints are created and solved, a variable is considered sensitive if
  its sensitivity value is at least that of the value assigned to sensitive
  information.

  
  For this to function as expected, just looking at the operands in the IR is
  not enough. If information flows from pointer operands, the points-to analysis
  is recruited to more precisely track the true source and sink data operands.
  Pointers and the target data are represented in a graph in the points-to
  analysis. Any operand which is a pointer, should have a node in the points-to
  graph and then the node itself can be used to help identify and properly
  constrain the data elements that are used in the computation of each variable.

  In the baseline analysis, each time a operand operated from a base address
  there was one memory element which represented the data from that pointer. If
  any field which is based from that pointer was computed from a sensitive
  value, all fields represented by that pointer would be considered sensitive.
  This is where the major modification needed to be made. Instead of one memory
  element representing all fields from a pointer, each pointer would need to map
  to be able to be treated independently.

\begin{algorithm}
\lstinputlisting{examples/struct_ex.c}
\caption{Structure Inaccuracy}
\label{alg:inacc}
\end{algorithm}

  Improving field sensitivity helps in reducing the number of false positives.
  Without augmentation the analysis would over-estimate the number of affected
  values for any given confidential information flow. In this example a
  structure type is defined such that it has two fields contained within it,
  both integers. The only variable that is set to be confidential is
  \texttt{key}. Lines 6,7,8 are all just declarations, at this point no
  variables are confidential apart from \texttt{key}. Line 10, the field of
  \texttt{a} in variable \texttt{u} is set to be the value of \texttt{key}. Here
  this means that there is a flow of information from the \texttt{key} variable
  to the \texttt{a} variable in the \texttt{u} data structure. Line 11, though
  \texttt{b} resides in the same data structure, \texttt{u}, as \texttt{a},
  there has been no flow from between \texttt{key} and \texttt{b}. At this point
  the confidential variables are \texttt{key} and $u.a$. As for the branches,
  line 13 is decided based on confidential data but line 15 is not. Line 14 is
  considered if implicit flow is considered, because the state of \texttt{ct}
  change based on the value of \texttt{key}.

\begin{algorithm}
\lstinputlisting{examples/minimal_struct_ex.ll}
\caption{LLVM intermediate representation}
\label{alg:ab_ir}
\end{algorithm}


The intermediate representation that is analyzed is similar to the code shown in
Algorithm \ref{alg:ab_ir}. This is not the exact representation from LLVM, but
this does display the vital components that the improved analysis leverages to
increase sensitivity. This snippet shows the IR code to accomplish lines 10 and
11 from Algorithm \ref{alg:inacc}. First, the value of key is loaded into a
register \texttt{\%0}, creating a flow from the value of key to the register
\texttt{\%0}. Next there is an instruction which gets the address of field
\texttt{a} from the structure \texttt{u}, which is located at an offset of 0.
Finally, line 3 completes storing the value of key to the location pointed to by
\texttt{\%a}. The taint analysis then needs a way to figure out what
\texttt{\%a} actually points to, to properly track the flow. Since \texttt{u} is
essentially a base address from which the offsets for fields \texttt{a} and
\texttt{b} are calculated the points to analysis keeps track of that memory
location. When an operation occurs in that region of memory, like when we do a
store to \texttt{u.a} there the points to analysis is used to track the memory location
that is modified. In this case it is trivial because the base pointer never
changes, but in cases where the base pointer may be modified, the points to
analysis maintains accurate representation of memory modifications. With the
aid of the points-to analysis, the taint analysis can track what parts of
memory were computed from confidential data. The next constraint that is
generated is the flow from \texttt{\%0} to the memory location which represents
\texttt{u} and ideally should only make \texttt{a} confidential. In the original
analysis, all structures were treated as one entity and flow from a confidential
source to a memory location was less precise. Precision in this context meant
that even though \texttt{a} and \texttt{b} are computed from the same base pointer and memory
location, only \texttt{a} has been affected by the confidential value. The original
analysis would instead have two values which are confidential \texttt{u.a} and \texttt{u.b}
due to this precision issue. As a result, the number of false positives are
lower with the analysis with increased precision. For algorithm \ref{alg:inacc},
lines 10, 11, 13 and 15 have potentially been computed from the confidential
value \texttt{key}, when the more precise analysis reports lines 10 and 13 as lines
which are dependent on the confidential value. Lines 11 and 15 are reported due
to \texttt{u.b} and \texttt{u.a} both adding constraints on the same memory structure, but not
distinguishing which part of the structure. The way to address this issue is to
consider the offset from the base pointer, as seen in lines 2 and 4 of Algorithm
\ref{alg:ab_ir}, given by the operands \texttt{i31 0} and \texttt{i32 1}.


\subsection{Result Classification}

The analysis provides a list of source code lines which are tainted. The
reported lines can then be reviewed and sorted to rank them in order of
severity. Reviewing the lines consists of three stages, the first removes
error-handling results, the second sorts the remaining results into either a
high or low-risk class, and the final sorts the high-risk results by their
context.

Stage 1 is a filter which removes the error handling and input validation
results. For normal operation of a program, these results are checking sensitive
data but results in an early exit if the branch condition evaluates to true. In
algorithm \ref{alg:stage0_src}, the variable $N$ is tainted, so the branch is
vulnerable due to the condition beign depending on $N$ and the data held within
$N$. The result of this branch is not helpful in the case of an attacker who
wishes to learn the data in $N$, since as long as the input is valid, the
condition of this branch will always evaluate to false. The results that are
analyzed often have a pattern of the branch leading to some error/exit code as the
target of the branch. If this pattern is identified the vulnerable result is
eliminated from the list.

\begin{algorithm}
\caption{Validation Source Code}
\begin{lstlisting}
if( mbedtls_mpi_cmp_int( N, 0 ) <= 0 || ( N->p[0] & 1 ) == 0 )
    return( MBEDTLS_ERR_MPI_BAD_INPUT_DATA );
\end{lstlisting}
\label{alg:stage0_src}
\end{algorithm}

Stage 2 operates on the remaining results to sort them into high and low
risk categories. A high risk branch will be due to one of the operands being
either directly related to the sensitive data or derived from more than 1 bit of
sensitive data. Low risk branches, are branches which the operand is based on 1
bit of the key or the same operand could derived from some set of sensitive data
values.

In Algorithm \ref{alg:branch_risk_examples}, ep is a pointer to the senitive
data, the branch in the IR level is due to short-circuit evaluation of the
condition. The first operand to the branch is computed from the sensitive data,
but at most only reveals 1 bit and as such is classified as a low-risk result.


\begin{algorithm}
  \caption{High-Risk and Low-Risk Branches}
\begin{lstlisting}
#define MPN_NORMALIZE(d, n)  \
do {		                     \
  while( (n) > 0 ) {         \
    if( (d)[(n)-1] )         \
      break;	               \
    (n)--;	                 \
  }		                       \
} while(0)

MPN_NORMALIZE(ep, esize);
if (esize * BITS_PER_MPI_LIMB > 512)
  W = 5;
\end{lstlisting}
  \label{alg:branch_risk_examples}
\end{algorithm}


This is a snippet of some of the modular exponentiation code from Libgcrypt
1.8.2. The values pointed to by $ep$ are marked as tainted at the start of the
analysis. The result of analyzing this code is that lines 10 and 11 are
reported. Line 10 is the macro defined in the lines 1-8. Line 10 is reported due
to the macro having the \texttt{if} branch directly dependent on the data
pointed to by \texttt{ep}. Since the branch is directly dependent on the tainted
data that branch is considered high-risk. Line 11 however is a branch based on
\texttt{esize}, the value begin modified within the \texttt{MPN\_NORMALIZE}
macro. The value of \texttt{esize} is not unique to the tainted data, meaning there
exists a set of tainted values which may yield the same \texttt{esize}. If a
value is derived from tainted data, as \texttt{esize} is, but the result is not
unique to that instance of tainted data, it is considered low-risk.

Sometimes it is unclear whether or not the risk is high or
low, as a result these are considered to be high-risk results to be
conservative. Having sorted the result into high and low-risk categories,
further categorization can be done in Stage 3. Additionally, if there are a
large number of low-risk result, Stage 2 can be repeated, by appending the
low-risk variables to the whitelist.

Stage 3 specifically looks a the context of the high-risk results from Stage 2.
Each of the results will be classified as either a single branch, repeated
branch, controllable branch or in extreme cases both controllable and repeated.
Single branches are branches which are not within a loop. Repeated branches are
branches that are part of a loop definition or within the body of a loop.
Controllable branches are that which the branch result is both untrusted and
tainted. The purpose of sorting results this way is to be able to prioritize
which vulnerable branches to examine first when attempting to make the
application secure.


% The desired change would allow each field in a
% user-defined type to be treated independently. The major change made to allow
% this type of behavior was to change how constraints were generated for
% AbstractLocs. Instead of one constraint which represented the entire memory
% location, constraints could be generated based on the offset accessed during a
% particular instruction. GetElementPtrInst (GEP instructions) are the type of
% instructions that are used to access a particular offset in to a pointer type
% in LLVM. The analysis generates constraints by moving through the flow
% information of the program. When iterating through the flow of information,
% some changes were made to handle these GEP instructions values differently
% than other value types.

%#
%  Any time a GEP instruction was captured, code was added to calculate what
%  offset was being accessed. If this was the first time an instruction tried to
%  access that memory as represented by an AbstractLoc, constraint elements were
%  generated based on the type information for that node in memory. If it had
%  already been accessed, the constraint element corresponding to the offset from
%  the instruction was returned. Once the constraint element is identified, the
%  right and left hand constraint elements are joined to add a constraint rule to
%  the programs set of constraints. This allowed individual fields from within
%  the same memory node could be tracked independently and could potentially
%  reduce the number of false positives compared to the previous design. There
%  are some challenges with allowing many constraints elements per memory node.
%
%  First, the GEP instruction value itself in LLVM represents the offset as an
%  index, not a byte offset. Meaning that if two separate GEP instruction uses
%  and offset of 2, one could index 8 bytes into the memory and the other 2
%  bytes. So in the new design this is handled by inspecting the LLVM value in
%  conjunction with the type information from the memory node to find the correct
%  element.
%
%  This leads to the second issue, the memory representation used in the analysis
%  can occasionally not be accompanied by type information. In this case, the
%  design defaults to using the LLVM representation of whatever type is being
%  addressed, and assumes field widths and that data elements do not overlap.
% The last change was made to the files that set the tainted values in either the
%
% vulnerable branch pass or the tainted analysis pass. These files read the lines
% from two files to establish what is to be tainted and untrusted. The values are
% set to be tainted by iterating through the list of values for the program. If a
% type which is mapped to a memory location is to be tainted there is a function
% to get the offset for that piece of memory, and consequently the constraint
% element tied to it. Previously, these offsets were calculated by going to the
% memory node type information, but changes were made here to properly get the
% correct offset when dealing with GEP instructions.maketitle
\section{Evaluation}
\subsection{Implementation}

Achieving the improved precision of the analysis, is done primarily through
handling LLVM's pointer instruction differently than other instructions. The
pointer instruction in the IR is the GetElementPtr instruction. This instruction
is used when computing an address from a base pointer. For arrays, for example
it has the index of the element and the bounds of the array if they are known.
For structures, the instruction contains the structure type which is being
addressed and the index of the field in the structure that is being referenced.
The baseline analysis lacked the ability to consider the index available in
these instructions. The points-to analysis had type and offset
information for the targets of the pointers. Improving the analysis was achieved
by i) creating the appropriate number of constraint elements for each node in
the points-to graph, ii) converting the field index from the GEP instruction to
a byte offset, and iii) constraining the correct elements based on
the GEP instruction and data type information.

\subsubsection{Constraint Element Generation}
When a pointer value is encountered for the first time, a node is created for it
in the points-to graph, and a set of constraint elements are created for that
data structure. The points-to analysis was left unmodified, so the focus will be
on the semantics of how constraint elements were changed. In the baseline, only
one constraint element represented any target of a pointer. In the improved
analysis, type information from the points-to analysis was used to generate the
appropriate number of constraint elements. In a stack allocated array, one
constraint element was created for each element in the array. For structures and
classes, each field is located at some offset away from the base address of the
structure and a constraint element was created for each field and mapped to
associated byte offset. This change enables the precision of the baseline to be
improved by correctly selecting the corresponding constraint element based on
the IR.

For any GEP instruction, constraint elements are generated using the type
information from the points-to analysis when available. Each constraint element
is created by iterating through the type information of the node from the
points-to analysis. For each type there is an associated byte offset, using this
offset and the size as reported by LLVM, a constraint element is created with
the corresponding starting byte offset and the ending byte offset based on the
width of the field. It is necessary to account for the width and start location
of each field because there may be padding between fields of a structure.
Padding between fields may be present in the type information because of an
unused field in code. The points-to analysis will not have type information on
unused fields so it is important to account for these gaps. Another reason to
account for the padding is due to structures which are not aligned with each
field following sequentially after the previous field. When the type information
is not available one constraint element is created for the whole node so at
worst the improved analysis will be equivalent to that of the baseline.

Algorithm \ref{alg:generate_conselem} shows how the points-to analysis is
leveraged to create constraint elements for the proper byte offsets and widths.
Let \texttt{s} be the LLVM representation of the structure type referenced by
the GEP instruction, and \texttt{node} be the node in the memory graph provided
by the points-to analysis. The algorithm works by retrieving the graph in which
the node resides and then retrieving the data layout as specified by the
compiler for that graph. The data layout contains information like the field
lengths of each type and the alignment within a data structure. The structure
type itself \texttt{s} does not have the alignment and padding information, only
the types of each field within the data structure. The data layout is used to
retrieve the structure layout of the structure type from the GEP instruction.
The structure layout is vital because it allows the index from the GEP
instruction to be converted to a byte offset within the data structure. The
data layout also provides the size of each type, so that the constraint element
is created at the corrects starting and ending offset.
\begin{algorithm}
  \caption{Creating constraint elements for each field in a type}
  \label{alg:generate_conselem}
  \lstinputlisting{examples/createConsElem.cpp}
\end{algorithm}

The elements created using this strategy enable the analysis to be more precise.
GEP instructions can also be used to index arrays, so that is handled by
computing the number of elements that array may hold. For stack arrays that
number is known, but for heap arrays the number is variable. For stack arrays
each, since the number of elements is known, one constraint element is created
per array element. For heap arrays, one constraint element is created for the
entire array.

Similarly, if type information is unavailable then a conservative approach is
used. One constraint element is created for the entire data structure. When
the information is available, the analysis is able to be improved by
constraining the correct constraint element for each instruction. When that
information is unavailable, the improved analysis is unable to be more precise
than the baseline analysis.

\subsubsection{Constraining Operands}
Each time an instruction where information flow exists, constraints are
generated between the operands of that instruction. The idea of this is simple,
given a set of constraint elements select the ones which are used in the
operation and constrain them as necessary. The baseline analysis handled
individual variables in a precise manner as long as the variable was not a
structure or array. The baseline analysis had no way of identifying the offset
from a GEP instruction, and even if it did, the mapping between a memory node to
a constraint element was one-to-one. In the previous section, the mapping from
memory node to constraint element was made to be one-to-many. As one node may
represent multiple fields which together make the structure or array. It is now
necessary to pick the correct element from this one-to-many mapping when
generating the constraint for each instruction. Again, this is done by analyzing
the GEP instruction.

The last operand of a GEP instruction is the index of the field within the data
structure. Each constraint element for a memory node is created for a byte
offset range, so it is necessary to correctly calculate the byte offset from the
field index provided by the GEP instruction. This is very similar to the
strategy used to generate the constraints at the correct byte locations. Given
the structure type, field index and node from the points-to analysis, the offset
is calculated as shown in Algorithm \ref{alg:find_offset}. Given the node, the
data layout and associated structure layout can be found and then the index can
be converted to a byte offset.

\begin{algorithm}
  \caption{Calculating byte offset from index}
  \label{alg:find_offset}
  \lstinputlisting{examples/findoffset.cpp}
\end{algorithm}

If the GEP instruction is in reference to an array access, then the offset is
just calculated using the size of the element type and the index. If no type
information exists for the node, only a single constraint element would exist
for the node, so that is the constraint which is selected to be constrained. 

\subsection{Benchmarks}
% list of the tests

\subsection{Case Study}

\section{Related Works}

\section{Conclusion}

\end{document}
